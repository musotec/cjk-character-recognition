Currently, character classification for Chinese, Japanese, and Korean (CJK) languages has been very limited. In real-time classification, results have since been improved from prior SVM implementations with convolutional neural networks (CNN). Although referred to here as "CJK" languages/characters the term is used as a collective for the logographic characters used in Hong Kong, Taiwan, and Vietnam in addition to the aforementioned countries. Prior research has generally treated each Chinese character as a whole, ignoring the internal two-dimensional structure within each character. The most recent state-of-the-art work has started to incorporate structure, but only achieves 40.82\% accuracy on unseen handwritten characters, despite 96.66\% accuracy in recall\cite{denseran}. Printed Chinese character identification using character metadata pushes the zero-shot performance to 56.6\%\cite{multi-attribute-recognition}, but is unable to translate the gain to handwriting. Online recognition used in input method editors (IME) relies on temporal data\cite{online-handwriting}, making it ineffective in OCR or when stroke order is incorrect.

In other computer vision classification tasks, recognition across over 9,000 classes has been achieved, in the YOLOv2 architecture\cite{yolo}. Similarly, the top three approaches are based on GoogLeNet \cite{hccr-googlenet}, ResNet \cite{multi-attribute-recognition}, and DenseNet \cite{denseran}; convolutional structures similar in nature. So it is unsurprising these three approaches have comparable rates around 96\% on the 3,755 class size of the GB2312-80 level-1 character set. This is due to reliance on the level-1 set of the CASIA handwriting database\cite{casia-handwriting-db}.

Despite this, accurate classification on the over 40,000 characters needed for complete modern day coverage in present day Chinese is a problem that has yet to be solved.

%The current state-of-the-art method in both zero-shot and recall, DenseRAN\cite{denseran}, utilizes attention, but can fail easily if the wrong path in attention is taken. Thus, the intuition of this proposal is that a bidirectional and hierarchical attention mechanism is necessary within the Chinese, and by extension, CJK character classification task.
%
%
%Bidirectional graph traversal allows for an efficient and correct encoding sequence of characters to be built. This intuition serves as the basis of the structure necessary to achieve performance in the zero-shot case, as well as recognition across the over 40,000 classes of Chinese logographic characters of modern use.
