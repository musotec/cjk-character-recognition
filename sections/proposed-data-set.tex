Following the work of WordNet\cite{wordnet} and ImageNet\cite{imagenet}, I am proposing the combination of data into a network that will capture links between logographic characters on a level not previously accumulated. This data set would capture character variants, constituents, positions, phonemes, translations, stroke orderings, and various encoding representations. All of these are already available for most characters, and are derivable from existing data. However, nothing has been amassed capturing the relationship between child radicals and parent compounds. The beauty of these writing systems comes from the interconnected nature of the ideographs, rebuses, associative and phono-semantic compounds that comprise the written languages.

The modularity of the three networks of this proposed architecture allows for future fine tuning. The Decoder can be tuned to answer questions only answerable by relationships between data within this set. Given the phono-semantic nature of the characters, this can be extended to phonetic outputs in Pinyin or Hiragana. However, it may be possible that the creation of a concept graph for graphemes behind the semantic portion of the characters can provide benefits in predictive input for IMEs, sentiment analysis, or even previously undiscovered insights on ancient writings of the Oracle bone script. It is my hope that such a network can provide sources for answers to these problems.
