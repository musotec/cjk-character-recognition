Following the work of WordNet\cite{wordnet} and ImageNet\cite{imagenet}, I am proposing the combination of data into a network that will capture links between logographic characters on a level not previously accumulated. This data set would capture character variants, constituents, positions, phonemes, translations, stroke orderings, and various encoding representations. All of these are already available for most characters, and are derivable from existing data. However, nothing matching the open nature of ILSVRC\cite{imagenet} has been amassed here. While there are datasets that capture the relationship between child radicals and parent compounds \cite{zhongwen}. However, the data is obfuscated, as it is behind images and the reverse relationships are not shown; limitations of the original book publication.

For example, the entry of \href{http://zhongwen.com/d/190/x199.htm}{學} has an absence of the character \textbf{子}.
However, the character \textbf{學} is included within the entry of \href{http://zhongwen.com/d/164/x108.htm}{子}. 
The entries also flatten many single node chains when referencing, at the cost of the important ideograph composition information.
While appropriate in forward query, the information needed for the reverse traversal of the bidirectional focus is lost.

\newpage
% New page to separate the feels from the reals.
 The beauty of these writing systems comes from the interconnected nature of the ideographs, rebuses, associative and phono-semantic compounds that comprise the written languages.
 Over all the modularity of the three components of this proposed architecture allows for extensibility in fine tuning.
 The Decoder can be tuned to answer questions answerable by relationships between radicals positionings, while the Encoder could be tuned to produce phonetic readings in Pinyin or Katakana. 
 With additional work on the dataset organization, it is possible that the resulting concept graph and semantic network can provide a deeper insight on the language level.
 These may be in predictive input for IMEs, improved sentiment analysis, or even advancements on ancient logoraphic interpretations in the writings of the Oracle bone script.
