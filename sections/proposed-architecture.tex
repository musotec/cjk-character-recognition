The problem of CJK character classification can be simplified by utilizing the structure within the characters. In order to tackle the handwritten recognition task effectively, we must simplify the problem. The approach outlined here is designed with goals of future extensibility through transfer learning. Each section describes a module that can be individually trained to achieve the end goal of offline CJK character recognition.

\subsection{Observe - Convolutional Handwriting Classifier}
The input must break the handwritten character into the constituent radicals that compose it. This task is similar to the object detection and classification problem. When performing detection in regions within a handwritten CJK character, the end goal is to group strokes belonging to each component within a compound character at the current depth. This can be performed on multiple feature grids by the multi-headed attention, providing resolution granularity during constituent observation. For an input, the network will return a feature vector of up to \(n\) constituents, $\vec{c}_n = <\vec{O}_0, \vec{O}_1, ... \vec{O}_{n-1}>$ at the current depth. The tuple $\vec{O}_n = (p_n,r_n)$ is the observed constituent at the current level, containing a position and radical encoding respectively.

The radical result, $r_n$ will be marked by \textbf{?} if the constituent is unrecognized, signaling to the Transformer that additional depth is needed in a particular region to achieve tokenization by the Encoder.

\subsection{Encode - Attention Based Character Tokenization}
The second network structure is the first half of a Transformer, prevalent in linguistic models of natural language processing\cite{transformers}. The Encoder uses the Observer network as a means of classification to tokenize a character into a sequence of embeddings. The embedding sequence leads to higher order classification at the Unicode character level, additionally driving the encoder-decoder attention.

Following the attention mechanism of Transformer, , attention can be given areas for use by the Observer network. While the linguistic case is unidimensional over a series in text, the lexicographic case mandates two dimensions to span the character space as shown in Figure \ref{fig:attention} and \ref{fig:encoder}.

% figure of observations
\begin{figure*}[h]
    \begin{center}
        \definecolor{bluee}{HTML}{0022ff}
\definecolor{orangee}{RGB}{233,127,2}
\definecolor{yellowe}{RGB}{208,162,0}
\definecolor{rede}{HTML}{ff2200}
\definecolor{greene}{RGB}{138,155,15}

\definecolor{metac}{RGB}{56,62,64}
\definecolor{dkbluee}{RGB}{0,62,114}

%\begin{document}
\begin{tikzpicture}

    % input rectangle
    \draw (0,0) node[scale=5, anchor=center, color=black!70] {学};
	\node[color=greene, rectangle, draw, thick, minimum width=5em, minimum height=5em] (input) at (0, 0) {};


    \node[color=bluee, rectangle, draw, thick, minimum width=4.55em, minimum height=3em] (A) at (0, -0.75em) {};
    \node[color=orangee, rectangle, draw, thick, minimum width=4.725em, minimum height=2em] (B) at (0, 1.3em) {};

    \node[color=yellowe, rectangle, draw, dashed, thick, minimum width=4.6em, minimum height=1.1em] (C) at (0, 0.9em) {};
    \node[color=rede, rectangle, draw, dashed, thick, minimum width=3.65em, minimum height=1em] (D) at (0, 1.7em) {};


	\node[color=greene!60!black, below=0em of input] (l1) {\emph{input}};

    \begin{scope}[minimum width=4em]
        \fontsize{16}{16}
    	\node[dkbluee, rectangle, draw, above right=-1.5em and 3em of input, thick]  (Ob) {${\vec{o}_{\tikz\pic[scale=0.5,color=dkbluee] {top};{\textbf{?}}}}$};

    	\node[dkbluee, rectangle, draw, above=0em of Ob, thick]  (Oa)  {${\vec{o}_{\tikz\pic[scale=0.5,color=dkbluee] {bottom};{\normalsize \textbf{子}}}}$};
    	\node[dkbluee, rectangle, draw, below=0em of Ob, thick]  (Oc)  {${\vec{o}_{\tikz\pic[scale=0.5,color=dkbluee] {top};{\normalsize \textbf{⺍}}}}$};
    	\node[dkbluee, rectangle, draw, below=0em of Oc,thick]  (Od)  {${\vec{o}_{\tikz\pic[scale=0.5,color=dkbluee] {bottom};{\normalsize \textbf{冖}}}}$};
    \end{scope}

    \node[dkbluee, left=0em of Oa] (la) {$\tikz{\pic[scale=0.75, color=bluee] at (0,0) {bottom};}$};
	\node[dkbluee, left=0em of Ob] (lb) {$\tikz{\pic[scale=0.75, color=orangee] at (0,0) {top};}$};
	\node[dkbluee, left=0em of Oc] (lc) {$\tikz{\pic[scale=0.75, color=rede] at (0,0) {top};}$};
	\node[dkbluee, left=0em of Od] (ld) {$\tikz{\pic[scale=0.75, color=yellowe] at (0,0) {bottom};}$};

	\node[dkbluee, below=0em of Od] (lr) {\emph{observations}};

    \draw[bluee,-stealth, thick] (A) -- (la);
	\draw[orangee,-stealth, thick] (B) -- (lb);
	\draw[rede,-stealth, thick, dashed] (C) -- (lc);
	\draw[yellowe,-stealth, thick, dashed] (D) -- (ld);

	\node[draw, dkbluee, thick, right=20em of input] (O) {
        $\tikz{
            \pic[scale=0.75, color=dkbluee] at (0,0) {bottom};
            \node at (0.3,0.175) {\textbf{子}};
            \pic[scale=0.75, color=dkbluee] at (.8,0) {top};
            \pic[scale=0.75, color=dkbluee] at (1.3,0) {top};
            \node at (1.6,0.175) {\textbf{⺍}};
            \pic[scale=0.75, color=dkbluee] at (2.1,0) {bottom};
            \node at (2.4,0.175) {\textbf{冖}};
        }$
    };
	\node[dkbluee, below=0em of O] {\emph{embedding sequence}};

	\draw[dkbluee, densely dashed, very thick] (Oa.north east) -- (O.north west);
	\draw[dkbluee, densely dashed, very thick] (Od.south east) -- (O.south west);

	\fill [opacity=0.2, dkbluee] (Oa.north east) -- (O.north west) -- (O.south west) -- (Od.south east) -- cycle;

	% invisible nodes for relationship arc drawings
	\node[right=14.75em of input, inner sep=0em] (dum1) {};
	\node[right=1.5em of Ob, inner sep=0em] (dum2) {};
	\node[right=1.5em of Oc, inner sep=0em] (dum3) {};

	\draw[dkbluee, densely dotted, very thick] (Oa.east) edge[bend left=60] (Ob.east);
	\draw[dkbluee, densely dotted, very thick] plot [smooth, tension=1.5] coordinates { (Oa.east) (dum2) (Oc.east)};
	\draw[dkbluee, densely dotted, very thick] plot [smooth, tension=1.5] coordinates { (Oa.east) (dum1) (Od.east)};
	\draw[dkbluee, densely dotted, very thick] (Ob.east) edge[bend left=60] (Oc.east);
	\draw[dkbluee, densely dotted, very thick] plot [smooth, tension=1.5] coordinates { (Ob.east) (dum3) (Od.east)};
	\draw[dkbluee, densely dotted, very thick] (Oc.east) edge[bend left=60] (Od.east);

	\node[metac,rectangle, thick, align=center, draw, below left=3em and -4em of Od, text width=13.5em] (Q) {\texttt{"What order of constituent logograms encodes this handwritten input?"}};
	\node[metac,below=0em of Q] (ql) {\emph{encoder network}};

	\path[metac,-stealth, dashed, thick] (Q.east) edge[bend right] (6.2, -0.7);

	\node[dkbluee] at (7, 0) (EN){\textbf{\emph{BERT}}};

\end{tikzpicture}
%\end{document}
        \caption[Encoder-Observer interaction]{The interactions of the Encoder network.
            The encoder uses it's attention to query the Observer for observations in a given region.
            The observations are used to create an embedding sequence for the Decoder.
            The attention mechanism is then driven in jointly by both the encoder and the decoder's processing of the token sequence. }
        \label{fig:encoder}
    \end{center}
\end{figure*}

The Encode network needs only to provide enough focus to obtain a sequence of encoded inputs capable of returning the correct Unicode character by the Decoder. This in practice makes execution time much faster than the $O(n^2)$ theoretical bound of a Transformer. Time is saved by avoiding extraneous and potentially misleading classifications, which can be seen in Figure \ref{fig:biang}. Processing only core constituents solves the misalignment between focus order and stroke order. This also improves performance in characters with variants by resolving before attention is spent on the differing features.

\begin{figure*}
    \begin{center}
        \definecolor{col1}{RGB}{255,0,0}
\definecolor{col2}{RGB}{0,255,0}
\definecolor{col3}{RGB}{0,0,255}

\newcommand{\SinCos}[4]{
    \begin{scope}[xscale=(5/#2)/2]
        \draw[color=#1!80,#3] (-1,0) sin (0,1) cos (1,0) sin (2,-1) cos (3,0) sin (4,1) cos (5,0) sin (6,-1) cos (7,0) sin (8,1) cos (9,0) sin (10,-1) cos (11,0) sin (12,1);
        \draw[color=#1!80!black,densely dash dot,xshift=28pt,#4] (-1,0) sin (0,1) cos (1,0) sin (2,-1) cos (3,0) sin (4,1) cos (5,0) sin (6,-1) cos (7,0) sin (8,1) cos (9,0) sin (10,-1) cos (11,0) sin (12,1);

        %        absolute value version
        %        \draw[color=#1!60] (-1,0) sin (0,1) cos (1,0) sin (2,1) cos (3,0) sin (4,1) cos (5,0) sin (6,1) cos (7,0) sin (8,1) cos (9,0) sin (10,1) cos (11,0) sin (12,1);
        %        \draw[color=#1!60!black, xshift=28pt] (-1,0) sin (0,1) cos (1,0) sin (2,1) cos (3,0) sin (4,1) cos (5,0) sin (6,1) cos (7,0) sin (8,1) cos (9,0) sin (10,1) cos (11,0) sin (12,1);
    \end{scope}
}

\begin{tikzpicture}[scale=0.9]
\begin{scope}[xshift=-150pt]
%    \begin{scope}[yshift=40pt] % Draw grid lines
%        \draw[help lines, color=red, dashed] (0,0) grid (5,5);
%        \draw[help lines, color=green, dashed] (0,0) grid[step=5/3] (5,5);
%        \draw[help lines, color=blue, dashed] (0,0) grid[step=5/2] (5,5);
%    \end{scope}
%

    \begin{scope}[rotate around={-90:(6.4,0)}, yscale=-1]
        \draw[<-] (-0.2,0) -- (5.2,0) node[below,name=y] {$y$};
        \draw[<->] (0,-1.3) -- (0,1.3);
        \draw (0,1) -- (-0.1,1) node[above] {1};
        \draw (0,-1) -- (-0.1,-1) node[above] {-1};
        \draw[<->] (5,-1.3) -- (5,1.3);
        \clip (0,-1.1) rectangle (5,1.1);
        \SinCos{red}{5}{}{ultra thick}
        \SinCos{green}{3}{}{ultra thick}
        \SinCos{blue}{2}{}{ultra thick}
    \end{scope}

    \begin{scope}
        \draw[<-] (-0.2,0) -- (5.2,0) node[right,name=x] {$x$};
        \draw[<->] (0,-1.3) -- (0,1.3);
        \draw (0,1) -- (-0.1,1) node[left] {1};
        \draw (0,-1) -- (-0.1,-1) node[left] {-1};
        \draw[<->] (5,-1.3) -- (5,1.3);

        \clip (0,-1.1) rectangle (5,1.1);

        \SinCos{red}{5}{}{ultra thick}
        \SinCos{green}{3}{}{ultra thick}
        \SinCos{blue}{2}{ultra thick}{}
    \end{scope}

    \draw (6.4,0) circle (3.5em) node[name=circ, text width=3em]{};
    \draw[thick] (6.4,0) circle (1.5em) node[align=center,below]{};
    \node at ($(circ.south)-(0,1.5cm)$) {\textit{positional\ encoding}};
    \draw[thick] ($(circ.center)$) to [controls=+(90:1em)and +(90:1em)] ($(circ.center)-(1.5em,0)$);
    \draw[thick] ($(circ.center)$) to [controls=+(-90:1em)and +(-90:1em)] ($(circ.center)+(1.5em,0)$);

    % draw kanji
    \begin{scope}[line cap=round, line width=3, scale=0.5, yscale=-1.05, yshift=-395, yslant=0.25]
    \pgfkeys{/warp/.style={color=black}}
    \useasboundingbox(0,0) rectangle (38.1mm,38.1mm);

    %% Group layer1 --> top=True
    %% Group kvg:StrokePaths_05b66 --> top=False
    %% Group kvg:05b66 --> top=False
    %% Group kvg:05b66-g1 --> top=False
    %% Group kvg:05b66-g2 --> top=False
    %% path id="kvg:05b66-s1"
    %% path spec="m 29.5,17.25 c 3.5,3 6.5,7.25 7.75,9.75"
    \draw[/warp] (29.5mm,17.2mm)
    .. controls ++(3.5mm,3.0mm) and ++(-1.2mm,-2.5mm) .. ++(7.8mm,9.8mm)
    ;
    %% path id="kvg:05b66-s2"
    %% path spec="m 49,12 c 1.25,2 4.75,8.25 5.25,11.5"
    \draw[/warp] (49.0mm,12.0mm)
    .. controls ++(1.2mm,2.0mm) and ++(-0.5mm,-3.2mm) .. ++(5.2mm,11.5mm)
    ;
    %% path id="kvg:05b66-s3"
    %% path spec="m 75,11 c 0.25,1.75 -0.12,2.75 -0.75,4.25 -1.29,3.1 -4.25,7.38 -6.5,9.75"
    \draw[/warp] (75.0mm,11.0mm)
    .. controls ++(0.2mm,1.8mm) and ++(0.6mm,-1.5mm) .. ++(-0.8mm,4.2mm)
    .. controls ++(-1.3mm,3.1mm) and ++(2.2mm,-2.4mm) .. ++(-6.5mm,9.8mm)
    ;
    %% Group kvg:05b66-g3 --> top=False
    %% path id="kvg:05b66-s4"
    %% path spec="M 21.25,33.75 C 21.13,38.5 19.25,46.25 17.5,50"
    \draw[/warp] (21.2mm,33.8mm)
    %%%% Warning: check controls
    .. controls (21.1mm,38.5mm) and (19.2mm,46.2mm) .. (17.5mm,50.0mm)
    ;
    %% path id="kvg:05b66-s5"
    %% path spec="m 23.5,36.5 c 17,-1.62 42.38,-5.5 60,-5.75 9.5,-0.13 4.12,5.12 0,9"
    \draw[/warp] (23.5mm,36.5mm)
    .. controls ++(17.0mm,-1.6mm) and ++(-17.6mm,0.2mm) .. ++(60.0mm,-5.8mm)
    .. controls ++(9.5mm,-0.1mm) and ++(4.1mm,-3.9mm) .. ++(0.0mm,9.0mm)
    ;
    %% Group kvg:05b66-g4 --> top=False
    %% path id="kvg:05b66-s6"
    %% path spec="m 37.25,46.5 c 1,0.25 3.75,0.25 5.5,-0.25 1.75,-0.5 18.25,-4 20,-4 1.75,0 2.75,0.75 1,2.25 C 62,46 54.5,53.5 53,54.75"
    \draw (37.2mm,46.5mm)
    .. controls ++(1.0mm,0.2mm) and ++(-1.8mm,0.5mm) .. ++(5.5mm,-0.2mm)
    .. controls ++(1.8mm,-0.5mm) and ++(-1.8mm,0.0mm) .. ++(20.0mm,-4.0mm)
    .. controls ++(1.8mm,0.0mm) and ++(1.8mm,-1.5mm) .. ++(1.0mm,2.2mm)
    %%%% Warning: check controls
    .. controls (62.0mm,46.0mm) and (54.5mm,53.5mm) .. (53.0mm,54.8mm)
    ;
    %% path id="kvg:05b66-s7"
    %% path spec="m 50.75,55.75 c 4,8.75 7.18,24.67 1.75,38 -2.75,6.75 -7.75,1.25 -9.75,-2"
    \draw[/warp] (50.8mm,55.8mm)
    .. controls ++(4.0mm,8.8mm) and ++(5.4mm,-13.3mm) .. ++(1.8mm,38.0mm)
    .. controls ++(-2.8mm,6.8mm) and ++(2.0mm,3.2mm) .. ++(-9.8mm,-2.0mm)
    ;
    %% path id="kvg:05b66-s8"
    %% path spec="m 15.75,67.75 c 1.75,1 4.64,1.36 7.5,1 15.88,-2 44.43,-6.25 61.37,-5.5 2.5,0.11 4.72,0.25 6.39,1"
    \draw[/warp] (15.8mm,67.8mm)
    .. controls ++(1.8mm,1.0mm) and ++(-2.9mm,0.4mm) .. ++(7.5mm,1.0mm)
    .. controls ++(15.9mm,-2.0mm) and ++(-16.9mm,-0.8mm) .. ++(61.4mm,-5.5mm)
    .. controls ++(2.5mm,0.1mm) and ++(-1.7mm,-0.8mm) .. ++(6.4mm,1.0mm)
    ;
\end{scope}

    % add selection grid
    \begin{scope}[transform shape, line cap=round,scale=0.5, yshift=80]

    %\draw[help lines] (10mm,10mm) grid (100mm,100mm);

    \def\wgrid{100mm}

    \draw[help lines, red, dashed, thick] (0mm,0mm) grid [step=\wgrid/5] (\wgrid,\wgrid);
    \draw[help lines, green!80!black, dashed, thick] (0mm,0mm) grid [step=\wgrid/3] (\wgrid,\wgrid);
    \draw[help lines, blue, dashed, thick] (0mm,0mm) grid [step=\wgrid/2] (\wgrid,\wgrid);
    \clip (0mm, 0mm) rectangle (\wgrid, \wgrid);

    \def\gx{1}
    \def\gy{2}
    \def\gxx{5}
    \def\bx{1}
    \def\by{0}
    \def\rx{1}
    \def\ry{1}
    \def\rwidth{3}
    \newcommand{\dr}{\wgrid/10}
    \newcommand{\dg}{\wgrid/6}
    \newcommand{\db}{\wgrid/4}

    \newcommand{\AttentionGrid}[2]{
        \node (blue) [rounded corners=1pt, #1=blue!90!black, minimum height=2*\db, minimum width=2*\db, #2] at (\db+\bx*\db,\db+\by*\db) {};
        \node (red) [rounded corners=1pt, #1=red!90!black, opacity=0.8, minimum height=2*\rwidth*\dr, minimum width=2*\rwidth*\dr, #2] at (\dr + \rx*\dr + \rx*\rwidth*\dr,\dr + \ry*\rwidth*\dr + \ry*\dr) {};
        \node (green) [rounded corners=1pt, #1=green!80!black, minimum height=2*\dg, minimum width=2*\dg, #2] at (\gx*\dg,\dg+\gy*\dg) {};

        \node (green2) [rounded corners=1pt, #1=green!80!black, minimum height=2*\dg, minimum width=2*\dg, #2] at (\gxx*\dg,\dg+\gy*\dg) {};
    }

    \begin{scope}[transparency group, fill opacity=0.55, draw opacity=1]
        \begin{scope}[blend group=screen]
            \AttentionGrid{fill}{}
        \end{scope}
    \end{scope}
    %
    \begin{scope}[transparency group, opacity=0.8, every node/.style={fill=none}]
        \AttentionGrid{ultra thick,draw,color}{remember picture}
    \end{scope}
\end{scope}

\end{scope}

\begin{scope}[yshift=0,xshift=0+175,on background layer]
    % Under Creative Commons Attribution licence 3.0
% (http://creativecommons.org/licences/by/3.0)
% Author: Tom Fuller
% Adapted from TikZ Cuboid Computation by: Florian Lesaint


%\definecolor{col1}{RGB}{255,0,0}
%\definecolor{col2}{RGB}{0,255,0}
%\definecolor{col3}{RGB}{0,0,255}


\newcommand{\generateRefPoints}[7]{
    % 1 = square width (-2cm)   = to be transformed by perspective computation
    % 2 = square height (-2cm)  = given position at 0
    % 3 = center_x (0)
    % 4 = center_y (0)
    % 5 = range (8cm)
    % 6 = above (1.5cm)
    % 7 = below (2cm)
    %
    % // todo: take array of lengths relative to prior
    % 1 = len1
    % 2 = len2
    % 3 = len3

    \pgfmathparse{#4+#2}    % center - height = underview
    \pgfmathresult \let\yunder\pgfmathresult;

    \pgfmathparse{#6+#4}    % center - height = overview
    \pgfmathresult \let\yabove\pgfmathresult;

    \pgfmathparse{#5-#3}    % center - width = base (positive)
    \pgfmathresult \let\px\pgfmathresult;

    \pgfmathparse{-#2}
    \pgfmathresult \let\ybelow\pgfmathresult;


    %    above = y_dist(a1 - p1); below = y_dist(a2 - p1)


    %    phi = atan(above/px)
    %    s*90 = 180 - 2*phi; solve for s
    %    x` = s * (above/below)
    %    y` = s * sin(phi)


    \pgfmathparse{atan2(\yabove,\px)}
    \pgfmathresult \let\phi\pgfmathresult;
    \pgfmathparse{(180 - 2*\phi)/90}    % obtain the side using law of sines, 90 degree angle avoids a sine call
    \pgfmathresult \let\side\pgfmathresult;

    % compute the new x from the above/below; simplified sin(90) = 1
    \pgfmathparse{\side * (\yabove / \ybelow)}
    \pgfmathresult \let\xprime\pgfmathresult;

    % TODO: FIX x-ratio by getting the ratio of the known height to the distance half way between P2
    % TODO: and then use as known side length for obtaining ratio to halfway to P1
    \pgfmathparse{#2/#1}
    \pgfmathresult \let\xratio\pgfmathresult;

    % now get a ratio to the perspective point from the relative x
    \pgfmathparse{(scalar(\px/1cm) + \xratio*\xprime)/scalar(\px/1cm)}  % TODO: devise a way to remove the units... cmon
    \pgfmathresult \let\ratio\pgfmathresult;


    %% Vanishing points for perspective handling
    \coordinate (P1) at (-#5, #6); % left vanishing point (To pick)
    \coordinate (P2) at (#5, #6);  % right vanishing point (To pick)

    %% (A1) and (A2) defines the 2 central points of the cuboid
    % TODO: use the parameters for center x and y
    \coordinate (A1) at (0em,0cm); % central top point (To pick)
    \coordinate (A2) at (0em,-#2); % central bottom point (To pick)

    % the picture always fits in half the range to make even
    \coordinate (X1) at ($(P2)!.5!(A1)$);
    \coordinate (X2) at ($(P2)!.5!(A2)$);

    %% (A3) to (A8) are computed given a unique parameter (or 2) .8
    % You can vary .8 from 0 to 1 to change perspective on left side
    \coordinate (A3) at ($(P1)!\ratio!(A2)$); % To pick for perspective
    \coordinate (A4) at ($(P1)!\ratio!(A1)$);

    % y prime is not needed, but is computed as such from law of sines
%    \pgfmathparse{\side * sin(\phi)}
%    \pgfmathresult \let\yprime\pgfmathresult;
%
%    % Draw debug point (needs yprime uncommented)
%    \draw[fill=black!80!transparent] (\xprime, \yprime) circle (2em) node[color=white] {\tiny \xratio};
}

% Generate the vanishing points for this concatenation
% TODO: clean up arguments for initial point generation
% TODO: factor out concatenated cube generation
\generateRefPoints{2cm}{2cm}{0}{0}{8cm}{1.5cm};

%%% Vanishing points for perspective handling
%\coordinate (P1) at (-8cm, 1.5cm); % left vanishing point (To pick)
%\coordinate (P2) at (8cm, 1.5cm);  % right vanishing point (To pick)
%
%%% (A1) and (A2) defines the 2 central points of the cuboid
%\coordinate (A1) at (0em,0cm); % central top point (To pick)
%\coordinate (A2) at (0em,-2cm); % central bottom point (To pick)
%
%\coordinate (X1) at ($(P2)!.5!(A1)$);
%\coordinate (X2) at ($(P2)!.5!(A2)$);
%
%% (A3) to (A8) are computed given a unique parameter (or 2) .8
% You can vary .8 from 0 to 1 to change perspective on left side
%\coordinate (A3) at ($(P1)!.8346!(A2)$); % To pick for perspective
%\coordinate (A4) at ($(P1)!.8346!(A1)$);

% You can vary .8 from 0 to 1 to change perspective on right side
\coordinate (A7) at ($(P2)!.9375!(A2)$);
\coordinate (A8) at ($(P2)!.9375!(A1)$);

%% Automatically compute the last 2 points with intersections
\coordinate (A5) at
(intersection cs: first line={(A8) -- (P1)},
   second line={(A4) -- (P2)});
\coordinate (A6) at
(intersection cs: first line={(A7) -- (P1)},
   second line={(A3) -- (P2)});


\coordinate (B1) at ($(P2)!.995!(A8)$);
\coordinate (B2) at ($(P2)!.995!(A7)$);

\coordinate (B3) at ($(P2)!.995!(A6)$);
\coordinate (B4) at ($(P2)!.995!(A5)$);

\coordinate (B7) at ($(P2)!.775!(A2)$);
\coordinate (B8) at ($(P2)!.775!(A1)$);


\coordinate (B5) at
  (intersection cs: first line={(B8) -- (P1)},
            second line={(A5) -- (P2)});

\coordinate (B6) at
  (intersection cs: first line={(B7) -- (P1)},
            second line={(A6) -- (P2)});



\coordinate (C1) at ($(P2)!.995!(B8)$);
\coordinate (C2) at ($(P2)!.995!(B7)$);

\coordinate (C3) at ($(P2)!.995!(B6)$);
\coordinate (C4) at ($(P2)!.995!(B5)$);

\coordinate (C7) at ($(P2)!.625!(B2)$);
\coordinate (C8) at ($(P2)!.625!(B1)$);


\coordinate (C5) at
(intersection cs: first line={(C8) -- (P1)},
second line={(B5) -- (P2)});

\coordinate (C6) at
(intersection cs: first line={(C7) -- (P1)},
second line={(B6) -- (P2)});

%%%%%%%%%%%%%%%%%%%%%%
%%%%%%%%%%%%%%%%%%%%%% Start filling

\fill[col1!70!black] (C2) -- (C3) -- (C6) -- (C7) -- cycle; % bottom face
\fill[col1!50] (C3) -- (C4) -- (C5) -- (C6) -- cycle;  % far left face
\fill[col1!30,opacity=0.5] (C5) -- (C6) -- (C7) -- (C8) -- cycle; % far right face
\draw[thick,dashed] (C5) -- (C6);
\draw[thick,dashed] (C3) -- (C6);
\draw[thick,dashed] (C7) -- (C6);
\fill[col1,opacity=0.7] (C1) -- (C8) -- (C7) -- (C2) -- cycle; % face 1
\node at (barycentric cs:C1=1,C8=1,C7=1,C2=1) {}; % \tiny c1};
\fill[col1,opacity=0.3] (C1) -- (C2) -- (C3) -- (C4) -- cycle; % c2
\fill[col1,opacity=0.7] (C1) -- (C4) -- (C5) -- (C8) -- cycle; % c5
\node (t1) [anchor=north] at (barycentric cs:C1=1,C4=1,C5=1,C8=1) {};
\draw[thick] (C1) -- (C2);
\draw[thick] (C3) -- (C4);
\draw[thick] (C7) -- (C8);
\draw[thick] (C1) -- (C4);
\draw[thick] (C1) -- (C8);
\draw[thick] (C2) -- (C3);
\draw[thick] (C2) -- (C7);
\draw[thick] (C4) -- (C5);
\draw[thick] (C8) -- (C5);

%\fill[green!70!black] (B2) -- (B3) -- (B6) -- (B7) -- cycle;
%\fill[green!50!black] (B3) -- (B4) -- (B5) -- (B6) -- cycle;
%\fill[green!30] (B5) -- (B6) -- (B7) -- (B8) -- cycle; % far face

\fill[col2!70!black] (B2) -- (B3) -- (B6) -- (B7) -- cycle;
\fill[col2!50] (B3) -- (B4) -- (B5) -- (B6) -- cycle;
\fill[col2!30,opacity=0.5] (B5) -- (B6) -- (B7) -- (B8) -- cycle; % far right face

\draw[thick,dashed] (B5) -- (B6);
\draw[thick,dashed] (B3) -- (B6);
\draw[thick,dashed] (B7) -- (B6);

\fill[col2,opacity=0.7] (B1) -- (B8) -- (B7) -- (B2) -- cycle; % face 1
\node at (barycentric cs:B1=1,B8=1,B7=1,B2=1) {}; % \tiny b1};

\fill[col2,opacity=0.3] (B1) -- (B2) -- (B3) -- (B4) -- cycle; % b2
%\node at (barycentric cs:B1=1,B2=1,B3=1,B4=1) {\tiny b2};

\fill[col2,opacity=0.7] (B1) -- (B4) -- (B5) -- (B8) -- cycle; % f5
\node (t2) [anchor=north] at (barycentric cs:B1=1,B4=1,B5=1,B8=1) {};


\draw[thick] (B1) -- (B2);
\draw[thick] (B3) -- (B4);
\draw[thick] (B7) -- (B8);
\draw[thick] (B1) -- (B4);
\draw[thick] (B1) -- (B8);
\draw[thick] (B2) -- (B3);
\draw[thick] (B2) -- (B7);
\draw[thick] (B4) -- (B5);
\draw[thick] (B8) -- (B5);

%%%
%%%

%% Possibly draw back faces
\fill[col3!70!black,opacity=0.7] (A2) -- (A3) -- (A6) -- (A7) -- cycle; % face 6
%\node at (barycentric cs:A2=1,A3=1,A6=1,A7=1) {\tiny f6};

\fill[col3!50!black,opacity=0.7] (A3) -- (A4) -- (A5) -- (A6) -- cycle; % face 3
%\node at (barycentric cs:A3=1,A4=1,A5=1,A6=1) {\tiny f3};

\fill[col3!30,opacity=0.5] (A5) -- (A6) -- (A7) -- (A8) -- cycle; % face 4
%\node at (barycentric cs:A5=1,A6=1,A7=1,A8=1) {\tiny f4};

\draw[thick,dashed] (A5) -- (A6);
\draw[thick,dashed] (A3) -- (A6);
\draw[thick,dashed] (A7) -- (A6);


%% Possibly draw front faces

\fill[col3,opacity=0.7] (A1) -- (A8) -- (A7) -- (A2) -- cycle; % face 1

\fill[col3!70!black,opacity=0.7] (A1) -- (A2) -- (A3) -- (A4) -- cycle; % f2

\fill[col3,opacity=0.7] (A1) -- (A4) -- (A5) -- (A8) -- cycle; % f5

\node at (barycentric cs:A1=1,A8=1,A7=1,A2=1) {};%\tiny a1};

\node at (barycentric cs:A1=1,A2=1,A3=1,A4=1) {};%\tiny a2};

\node (t3) [anchor=north] at (barycentric cs:A1=1,A4=1,A5=1,A8=1) {};

%% Possibly draw front lines
\draw[thick] (A1) -- (A2);
\draw[thick] (A3) -- (A4);
\draw[thick] (A7) -- (A8);
\draw[thick] (A1) -- (A4);
\draw[thick] (A1) -- (A8);
\draw[thick] (A2) -- (A3);
\draw[thick] (A2) -- (A7);
\draw[thick] (A4) -- (A5);
\draw[thick] (A8) -- (A5);

% Draw debug points
%\foreach \i in {1,2,...,8} {
%    \draw[fill=black] (A\i) circle (0.25em) node[color=white] {\tiny \i};
%}
%
%\draw[fill=black] (P1) circle (0.1em) node[below] {\tiny p1};
%\draw[fill=black] (P2) circle (0.1em) node[below] {\tiny p2};

\end{scope}

\begin{scope}[yshift=64,xshift=260,z={(10:10mm)},x={(165:10mm)}]
    \begin{scope}[canvas is zx plane at y=0, scale=1.125]

    % Draw the Kanji stroke on the grid
    \begin{scope}[line cap=round, line width=3, scale=0.5, yscale=-1.05, yshift=-395, yslant=0.25]
    \pgfkeys{/warp/.style={color=black}}
    \useasboundingbox(0,0) rectangle (38.1mm,38.1mm);

    %% Group layer1 --> top=True
    %% Group kvg:StrokePaths_05b66 --> top=False
    %% Group kvg:05b66 --> top=False
    %% Group kvg:05b66-g1 --> top=False
    %% Group kvg:05b66-g2 --> top=False
    %% path id="kvg:05b66-s1"
    %% path spec="m 29.5,17.25 c 3.5,3 6.5,7.25 7.75,9.75"
    \draw[/warp] (29.5mm,17.2mm)
    .. controls ++(3.5mm,3.0mm) and ++(-1.2mm,-2.5mm) .. ++(7.8mm,9.8mm)
    ;
    %% path id="kvg:05b66-s2"
    %% path spec="m 49,12 c 1.25,2 4.75,8.25 5.25,11.5"
    \draw[/warp] (49.0mm,12.0mm)
    .. controls ++(1.2mm,2.0mm) and ++(-0.5mm,-3.2mm) .. ++(5.2mm,11.5mm)
    ;
    %% path id="kvg:05b66-s3"
    %% path spec="m 75,11 c 0.25,1.75 -0.12,2.75 -0.75,4.25 -1.29,3.1 -4.25,7.38 -6.5,9.75"
    \draw[/warp] (75.0mm,11.0mm)
    .. controls ++(0.2mm,1.8mm) and ++(0.6mm,-1.5mm) .. ++(-0.8mm,4.2mm)
    .. controls ++(-1.3mm,3.1mm) and ++(2.2mm,-2.4mm) .. ++(-6.5mm,9.8mm)
    ;
    %% Group kvg:05b66-g3 --> top=False
    %% path id="kvg:05b66-s4"
    %% path spec="M 21.25,33.75 C 21.13,38.5 19.25,46.25 17.5,50"
    \draw[/warp] (21.2mm,33.8mm)
    %%%% Warning: check controls
    .. controls (21.1mm,38.5mm) and (19.2mm,46.2mm) .. (17.5mm,50.0mm)
    ;
    %% path id="kvg:05b66-s5"
    %% path spec="m 23.5,36.5 c 17,-1.62 42.38,-5.5 60,-5.75 9.5,-0.13 4.12,5.12 0,9"
    \draw[/warp] (23.5mm,36.5mm)
    .. controls ++(17.0mm,-1.6mm) and ++(-17.6mm,0.2mm) .. ++(60.0mm,-5.8mm)
    .. controls ++(9.5mm,-0.1mm) and ++(4.1mm,-3.9mm) .. ++(0.0mm,9.0mm)
    ;
    %% Group kvg:05b66-g4 --> top=False
    %% path id="kvg:05b66-s6"
    %% path spec="m 37.25,46.5 c 1,0.25 3.75,0.25 5.5,-0.25 1.75,-0.5 18.25,-4 20,-4 1.75,0 2.75,0.75 1,2.25 C 62,46 54.5,53.5 53,54.75"
    \draw (37.2mm,46.5mm)
    .. controls ++(1.0mm,0.2mm) and ++(-1.8mm,0.5mm) .. ++(5.5mm,-0.2mm)
    .. controls ++(1.8mm,-0.5mm) and ++(-1.8mm,0.0mm) .. ++(20.0mm,-4.0mm)
    .. controls ++(1.8mm,0.0mm) and ++(1.8mm,-1.5mm) .. ++(1.0mm,2.2mm)
    %%%% Warning: check controls
    .. controls (62.0mm,46.0mm) and (54.5mm,53.5mm) .. (53.0mm,54.8mm)
    ;
    %% path id="kvg:05b66-s7"
    %% path spec="m 50.75,55.75 c 4,8.75 7.18,24.67 1.75,38 -2.75,6.75 -7.75,1.25 -9.75,-2"
    \draw[/warp] (50.8mm,55.8mm)
    .. controls ++(4.0mm,8.8mm) and ++(5.4mm,-13.3mm) .. ++(1.8mm,38.0mm)
    .. controls ++(-2.8mm,6.8mm) and ++(2.0mm,3.2mm) .. ++(-9.8mm,-2.0mm)
    ;
    %% path id="kvg:05b66-s8"
    %% path spec="m 15.75,67.75 c 1.75,1 4.64,1.36 7.5,1 15.88,-2 44.43,-6.25 61.37,-5.5 2.5,0.11 4.72,0.25 6.39,1"
    \draw[/warp] (15.8mm,67.8mm)
    .. controls ++(1.8mm,1.0mm) and ++(-2.9mm,0.4mm) .. ++(7.5mm,1.0mm)
    .. controls ++(15.9mm,-2.0mm) and ++(-16.9mm,-0.8mm) .. ++(61.4mm,-5.5mm)
    .. controls ++(2.5mm,0.1mm) and ++(-1.7mm,-0.8mm) .. ++(6.4mm,1.0mm)
    ;
\end{scope}

    \begin{scope}[transform shape, line cap=round,scale=0.5, yshift=80]

    %\draw[help lines] (10mm,10mm) grid (100mm,100mm);

    \def\wgrid{100mm}

    \draw[help lines, red, dashed, thick] (0mm,0mm) grid [step=\wgrid/5] (\wgrid,\wgrid);
    \draw[help lines, green!80!black, dashed, thick] (0mm,0mm) grid [step=\wgrid/3] (\wgrid,\wgrid);
    \draw[help lines, blue, dashed, thick] (0mm,0mm) grid [step=\wgrid/2] (\wgrid,\wgrid);
    \clip (0mm, 0mm) rectangle (\wgrid, \wgrid);

    \def\gx{1}
    \def\gy{2}
    \def\gxx{5}
    \def\bx{1}
    \def\by{0}
    \def\rx{1}
    \def\ry{1}
    \def\rwidth{3}
    \newcommand{\dr}{\wgrid/10}
    \newcommand{\dg}{\wgrid/6}
    \newcommand{\db}{\wgrid/4}

    \newcommand{\AttentionGrid}[2]{
        \node (blue) [rounded corners=1pt, #1=blue!90!black, minimum height=2*\db, minimum width=2*\db, #2] at (\db+\bx*\db,\db+\by*\db) {};
        \node (red) [rounded corners=1pt, #1=red!90!black, opacity=0.8, minimum height=2*\rwidth*\dr, minimum width=2*\rwidth*\dr, #2] at (\dr + \rx*\dr + \rx*\rwidth*\dr,\dr + \ry*\rwidth*\dr + \ry*\dr) {};
        \node (green) [rounded corners=1pt, #1=green!80!black, minimum height=2*\dg, minimum width=2*\dg, #2] at (\gx*\dg,\dg+\gy*\dg) {};

        \node (green2) [rounded corners=1pt, #1=green!80!black, minimum height=2*\dg, minimum width=2*\dg, #2] at (\gxx*\dg,\dg+\gy*\dg) {};
    }

    \begin{scope}[transparency group, fill opacity=0.55, draw opacity=1]
        \begin{scope}[blend group=screen]
            \AttentionGrid{fill}{}
        \end{scope}
    \end{scope}
    %
    \begin{scope}[transparency group, opacity=0.8, every node/.style={fill=none}]
        \AttentionGrid{ultra thick,draw,color}{remember picture}
    \end{scope}
\end{scope}
    \end{scope}

    \begin{scope}[canvas is zy plane at x=0]
        \node (head) at ($(blue.south)+(0.25,-0.25)$) [transform shape] {multi-head attention};
    \end{scope}

    \begin{scope}[z={(12.5:7mm)},canvas is zy plane at x=0]
        \node (conv) at ($(blue.south)+(0,-1.1)$) [transform shape] {{\footnotesize convolutional observation network}};
    \end{scope}

    \begin{scope}[z={(15:5mm)},canvas is zy plane at x=0]
        \draw[transform shape, ->] (head) -> (conv);
        \node (conc) at ($(blue.south)+(-1.5,-2)$) [transform shape] {concatenate};
    \end{scope}

\end{scope}
\begin{scope}[xshift=260,z={(26:10mm)},x={(165:15mm)}] %,y={(-155:10mm)}]
    \begin{scope}[canvas is zx plane at y=-200]
        \node at ($(B7.south)+(-0.5, -0.5)$) [transform shape] {encoder-decoder attention};
    \end{scope}
\end{scope}

\draw[->] (conv) -> (conc);

% TODO: ensure that these points exist on the concat-filters-3d
\begin{scope}[on background layer, out=-90,in=90,looseness=1.5, opacity=0.7]
    \draw[red!80!black, ultra thick, double distance=1pt] (red.mid) to (t1);
    \draw[green!80!black,ultra thick, double distance=1pt] (green.mid) to (t2);
    \draw[green!80!black,ultra thick, double distance=1pt] (green2.mid) to (t2);
    \draw[blue!80!black,ultra thick, double distance=1pt] (blue.mid) to (t3);
\end{scope}

\end{tikzpicture}

        \caption[Multi-head Positional Encoding of Transformer]{The positional encoding mechanism of the Transformers. On the left, input sinusoids are shown. The selected waves in bold are chosen arbitrarily from each sine and cosine pair. The waves of highest energy for a given input is selected to control the area of attention. On the right, the same attention selection for the three heads is shown being passed through the convolutional observer and concatenated. The concatenated result is then used by the Encoder and Decoder to determine future positions of attention.}
        \label{fig:attention}
    \end{center}
\end{figure*}

\subsection{Decode - Unicode Character Retrieval from Embedding Sequence}
The final portion of the network will process two encoded sequences in parallel, as part of the bidirectional approach seen in BERT\cite{bert}. By creating two regions of focus, the result can be reached faster, and potentially without using all the information. The main result from the decoder is in providing a single UTF-16 character (represented across a weighted probability vector of all Unicode CJK characters).

Since the decode network takes only a tokenized embedding sequence, the decoder can be trained quickly with ground truth data outside of the computer vision components. Fine tuning of Transformers in subsequent training sessions has been quite effective\cite{transformers}\cite{bert}. By training the fully connected output layers from the decoder, future questions can be answered, making this architecture extensible.

\subsection{Summary}
The Observer network will learn
\begin{itemize}
    \item "Are any CJK constituents located in this handwriting sample? Where?"
    \item "Is the constituent a known radical? should it be broken further?"
\end{itemize}
The Encode network will learn
\begin{itemize}
    \item "What does the position of one ideograph mean in relation to others?"
    \item "What region should get attention so a tokenized sequence can be given to the decoder?"
\end{itemize}
The Decode network will learn
\begin{itemize}
    \item "What Unicode CJK character does this sequence of radicals and positions represent?"
    \item "After what has been seen, what region should get attention next?"
\end{itemize}

The operational flow of the entire network is shown in Figure \ref{fig:use}.

\newpage
\newcommand\tab[1][1cm]{\hspace*{#1}}
\section{Overview}
\begin{figure*}[h]
    \begin{center}
        %\documentclass[crop, tikz]{standalone}
%\usepackage{tikz}
%\usepackage{amsmath}
%
%\usetikzlibrary{decorations.pathmorphing, positioning, backgrounds, arrows.meta, calc}

\definecolor{echoreg}{HTML}{ff4400}
\definecolor{echodrk}{HTML}{0044ff}

\tikzstyle{mynode} = [text=black, very thick,
    rectangle, inner sep=10pt, inner ysep=20pt]
\tikzstyle{fancytitle} =[text=black]


\tikzstyle{labelnode} = [text=black, very thick,
draw=none, font=\normalsize, outer sep=0pt]

\newcommand{\yslant}{0.5}
\newcommand{\xslant}{-0.6}

\begin{tikzpicture}[scale=0.58,every node/.style={minimum size=1cm},on grid]

	\node [mynode, scale=0.85] at (10.5, 0) (box){%
		\begin{minipage}{0.2\textwidth}

          \tikz[
              yshift=50,
              edge from parent path={(\tikzparentnode.south) -- (\tikzchildnode.north)},
              every child node/.style={color=black,scale=0.35,thick,outer sep=0pt,circle,minimum size=16pt,draw,column sep={1.5pt,between origins}},
              font=\Huge
          ] {

          % define the set of forward observed nodes
          \node (fA) [draw, thick, circle, scale=0.35, fill=echoreg!20] at (0,0) {A}; % 十 pin={0:\scriptsize{t}}
          \node (fB) [draw, thick, circle, scale=0.35, fill=echoreg!20] at (0,-1) {\textbf{?}};
          \node (fX) [draw, thin, circle, scale=0.35, fill=white] at (1,-1) {宀}; % 宀 (字)
          \pic [scale=0.9] at ($(fA.mid)-(0.225,0.155)$) {top};
          \node (fC) [draw, thick, circle, scale=0.35, fill=echoreg!40, double] at (0,-2) {C};
          \pic [scale=0.9] at ($(fC.mid)-(0.225,0.155)$) {top};
          \node (fD) [draw, dashed, thick, circle, scale=0.35, fill=white] at (1,-2) {D};
          \pic [scale=0.9] at ($(fD.mid)-(0.225,0.155)$) {bottom};

          % define the set of result nodes
          \begin{scope}[xshift=5,yshift=5,scale=1.1,
              every pin edge/.style={draw=none}]
              \node (rA) [draw, thick, circle, scale=0.35, fill=white, dashed] at (0,-4) {⺾}; % ⺾ (芥) 1st CUT [荐,菰 because bottom is compound, top is not]; also cuts 芥苓茶荅荼莟蒼蓉蔭蓼薈... after first observation without having to look at [top][bottom] to determine if it is 冖 or 𠆢. ⺾/𠆢 in this path is fully cut before observation of [top][top] because ALL possible bottom characters in the tree are not 子.

              \node (rB) [draw, thick, circle, scale=0.35, fill=white, pin={[pin distance=-15pt]below:\textsf{孛}}] at (1,-4) {十}; % 十 (孛/character)(學/school)
              \node (rC) [draw, thick, circle, scale=0.35, fill=echoreg!50!echodrk!50, double, pin={[name=p1,pin distance=-15pt]below:\textbf{\underline{学}}}] at (2,-4) {\textbf{⺍}}; % ⺍ (学/study)
              \node (rD) [draw, thick, circle, scale=0.35, fill=white, pin={[pin distance=-15pt]below:學}] at (3,-4) {𦥯}; % 𦥯 (學/school)
              \node[labelnode, above=-15pt, scale=1.15] at ($(p1.south)$) {\textbf{(f)}};
          \end{scope}
          % 學学孛字孕

%          \node (D) [draw, thick, circle, scale=0.35, fill=echoreg!40] at (2,-5) {D};

%  電子 (でんしゃ) means "electron" but literally translates to "electric child", 孛 combines "above" and "child"



          \begin{scope}[xshift=70]
              \node (bA) [draw, thick, circle, scale=0.35, fill=echodrk!20] at (0,0) {A};
              \node (bB) [draw, thick, circle, scale=0.35, fill=echodrk!20] at (0,-1) {\textbf{子}};
              \pic [scale=0.9] at ($(bA.mid)-(0.225,0.155)$) {bottom};
              \node (bC) [draw, thick, circle, scale=0.35, fill=echodrk!40] at (1,-1) {C};
              \pic [scale=0.9] at ($(bC.mid)-(0.225,0.155)$) {top};

              \node (bD) [draw, thick, circle, scale=0.35, fill=echodrk!40, double] at (0,-2) {D};
              \pic [scale=0.9] at ($(bD.mid)-(0.225,0.155)$) {top};

              \node (bE) [draw, thick, circle, scale=0.35, fill=white, dashed] at (1,-2) {E};
              \pic [scale=0.9] at ($(bE.mid)-(0.225,0.155)$) {bottom};
          \end{scope}

          \node[labelnode, scale=1.15] at ($(fD)!.5!(bD) - (0,15pt)$) {\textbf{(e)}};
          \draw[ultra thick] (fA) -- (fB);
          \draw[ultra thick] (fB) -- (fC);
          \draw[thin] (fA) -- (fX);
          \draw[densely dashed] (fB) -- (fD);

          \draw[ultra thick] (bA) -- (bB);
          \draw[ultra thick] (bB) -- (bC);
          \draw[ultra thick] (bC) -- (bD);
          \draw[densely dashed] (bC) -- (bE);

          \draw[densely dashed] (fC) -- (rA);
          \draw (fC) -- (rB);
          \draw[ultra thick] (fC) -- (rC);
          \draw (fC) -- (rD);

          \draw (bD) -- (rB);
          \draw[ultra thick] (bD) -- (rC);
          \draw (bD) -- (rD);

          }
    	\end{minipage}
	};

	\node[fancytitle, scale=0.8] at (box.north) {\bf Sparse Trie Traversal:};

	% Layer 2
	\begin{scope}[
		yshift=-250,
		every node/.append style={yslant=\yslant,xslant=\xslant},
		yslant=\yslant,xslant=\xslant,
        on background layer
	]
        \draw[black, dashed, thin] (0,0) rectangle (7,7);

        \begin{scope}[yshift=-20,xshift=40]
         \begin{scope}[yshift=40]
            \def\wgrid{50mm}
            \def\wbox{\wgrid/4}

            \draw[help lines, black!40] (0mm,0mm) grid [step=\wgrid/4] (\wgrid, \wgrid);

            % Draw nodes and labels on graph
            \begin{scope}[xshift=0.5*\wbox, yshift=0.5*\wbox]
                % nodes
                \node (a) [draw, thick, circle, scale=0.5, fill=echoreg] at (0*\wbox, 3*\wbox) {};
                \draw[fill=echodrk] (3*\wbox,0*\wbox) circle (.1);
                \node (b) [draw, thick, circle, scale=0.5, fill=echodrk] at (3*\wbox, 0*\wbox) {};
                \node (c) [draw, double, circle, scale=0.5, fill=echodrk!50!echoreg] at (2*\wbox, 2*\wbox) {};
                \node (d) [draw, densely dashed, circle, scale=0.5] at (1*\wbox, 1*\wbox) {};

                % top labels
                \pic[scale=0.45, color=black, anchor=south] at (3*\wbox-10,4*\wbox-10) {bottom};
                \node [scale=0.75,anchor=west] at (3*\wbox-10,4*\wbox-10) {\textbf{子}};
                \pic[scale=0.45, color=black, anchor=south] at (0*\wbox-10,4*\wbox-10) {top};
                \node [scale=0.75,anchor=west] at (0*\wbox-10,4*\wbox-10) {\textbf{?}};
                \pic[scale=0.45, color=black, anchor=south] at (1*\wbox-10,4*\wbox+3) {top};
                \pic[scale=0.45, color=black!80, anchor=south] at (1*\wbox-10,4*\wbox-13) {bottom};
                \node [scale=0.75,anchor=west] at (1*\wbox-10,4*\wbox-10) {\textbf{冖}};
                \pic[scale=0.45, color=black, anchor=south] at (2*\wbox-10,4*\wbox+3) {top};
                \pic[scale=0.45, color=black!80, anchor=south] at (2*\wbox-10,4*\wbox-13) {top};
                \node [scale=0.75,anchor=west] at (2*\wbox-10,4*\wbox-10) {\textbf{⺍}};

                % left labels
                \pic[scale=0.45, color=black, anchor=south] at (-1*\wbox,0*\wbox-10) {bottom};
                \pic[scale=0.45, color=black, anchor=south] at (-1*\wbox,3*\wbox-10) {top};
                \pic[scale=0.45, color=black, anchor=south] at (-1*\wbox,1*\wbox+3) {top};
                \pic[scale=0.45, color=black!80, anchor=south] at (-1*\wbox,1*\wbox-13) {bottom};
                \pic[scale=0.45, color=black, anchor=south] at (-1*\wbox,2*\wbox+3) {top};
                \pic[scale=0.45, color=black!80, anchor=south] at (-1*\wbox,2*\wbox-13) {top};
            \end{scope}
         \end{scope}
        \end{scope}

        % Draw edges between nodes
        \draw[-latex, thick, color=echoreg] (a) -- (c);
		\draw[-latex, thick, color=echodrk] (b) -- (c);
        \draw (a) edge [draw, -latex, thin, dashed, color=black!70] (d);
        \draw (c) edge [draw, -latex, thin, dashed, bend right=35, looseness=1.25, color=echoreg] (d);
        \draw (c) edge [draw, -latex, thin, dashed, bend right=-35, looseness=1.25, color=echodrk] (d);

        \fill[black] (1.75,0.35) node[right, scale=.7] {Bidirectional Attention};
	\end{scope}

    % Observation Layer
    \begin{scope}[
        yshift=-60,
        every node/.append style={yslant=\yslant,xslant=\xslant},
        yslant=\yslant,xslant=\xslant
        ]
        \fill[white,fill opacity=.75] (0,0) rectangle (7,7);
        \draw[black, dashed, thin] (0,0) rectangle (7,7);

        \begin{scope}[yshift=-20,xshift=20]
            \begin{scope}[yshift=40]
                \def\wgrid{50mm}
                \def\wbox{\wgrid/16}

                \draw[help lines, black!40] (0mm,0mm) grid [step=\wgrid/16] (\wgrid, \wgrid);
                % Create the pixel labelings for the observations
                \begin{scope}[transparency group, opacity=0.6]
                    \fill[fill=black!30] (0,0) rectangle (\wgrid, \wgrid);

                    \fill[fill=echoreg] (\wbox*2, \wbox*9) rectangle (\wbox*4, \wbox*13);
                    \fill[fill=echoreg] (\wbox*4, \wbox*15) rectangle (\wbox*13, \wbox*10);
                    \fill[fill=echoreg] (\wbox*12, \wbox*11) rectangle (\wbox*15, \wbox*8);

                    \fill[fill=blue] (\wbox*10, \wbox*7) rectangle (\wbox*15, \wbox*3);
                    \fill[fill=blue] (\wbox*5, \wbox*10) rectangle (\wbox*11, \wbox*8);
                    \fill[fill=blue] (\wbox*6, \wbox*10) rectangle (\wbox*11, \wbox*0);
                    \fill[fill=blue] (\wbox*2, \wbox*8) rectangle (\wbox*8, \wbox*4);

                    \node (obstop) [draw=none] at (\wgrid/2,\wbox*13) {};
                    \node (obsbot) [draw=none] at (\wgrid/2,\wbox*6) {};
                \end{scope}
            \end{scope}
        \end{scope}

        \fill[black] (3.75,0.35) node[right, scale=.7] {Observations};
    \end{scope}

	% Input Layer
	\begin{scope}[
		yshift=0,
		every node/.append style={yslant=\yslant,xslant=\xslant},
		yslant=\yslant,xslant=\xslant
	]
		\fill[white,fill opacity=.75] (0,0) rectangle (7,7);
		\draw[black, dashed, thin] (0,0) rectangle (7,7);

        \begin{scope}[yshift=-20,xshift=20]
            \begin{scope}[yshift=40]
                \def\wgrid{50mm}
                \def\wbox{\wgrid/16}

                \draw[help lines, black!40] (0mm,0mm) grid [step=\wbox] (\wgrid, \wgrid);

            \end{scope}

            \begin{scope}[line cap=round, line width=3, scale=0.5, yscale=-1.05, yshift=-395, yslant=0.25]
    \pgfkeys{/warp/.style={color=black}}
    \useasboundingbox(0,0) rectangle (38.1mm,38.1mm);

    %% Group layer1 --> top=True
    %% Group kvg:StrokePaths_05b66 --> top=False
    %% Group kvg:05b66 --> top=False
    %% Group kvg:05b66-g1 --> top=False
    %% Group kvg:05b66-g2 --> top=False
    %% path id="kvg:05b66-s1"
    %% path spec="m 29.5,17.25 c 3.5,3 6.5,7.25 7.75,9.75"
    \draw[/warp] (29.5mm,17.2mm)
    .. controls ++(3.5mm,3.0mm) and ++(-1.2mm,-2.5mm) .. ++(7.8mm,9.8mm)
    ;
    %% path id="kvg:05b66-s2"
    %% path spec="m 49,12 c 1.25,2 4.75,8.25 5.25,11.5"
    \draw[/warp] (49.0mm,12.0mm)
    .. controls ++(1.2mm,2.0mm) and ++(-0.5mm,-3.2mm) .. ++(5.2mm,11.5mm)
    ;
    %% path id="kvg:05b66-s3"
    %% path spec="m 75,11 c 0.25,1.75 -0.12,2.75 -0.75,4.25 -1.29,3.1 -4.25,7.38 -6.5,9.75"
    \draw[/warp] (75.0mm,11.0mm)
    .. controls ++(0.2mm,1.8mm) and ++(0.6mm,-1.5mm) .. ++(-0.8mm,4.2mm)
    .. controls ++(-1.3mm,3.1mm) and ++(2.2mm,-2.4mm) .. ++(-6.5mm,9.8mm)
    ;
    %% Group kvg:05b66-g3 --> top=False
    %% path id="kvg:05b66-s4"
    %% path spec="M 21.25,33.75 C 21.13,38.5 19.25,46.25 17.5,50"
    \draw[/warp] (21.2mm,33.8mm)
    %%%% Warning: check controls
    .. controls (21.1mm,38.5mm) and (19.2mm,46.2mm) .. (17.5mm,50.0mm)
    ;
    %% path id="kvg:05b66-s5"
    %% path spec="m 23.5,36.5 c 17,-1.62 42.38,-5.5 60,-5.75 9.5,-0.13 4.12,5.12 0,9"
    \draw[/warp] (23.5mm,36.5mm)
    .. controls ++(17.0mm,-1.6mm) and ++(-17.6mm,0.2mm) .. ++(60.0mm,-5.8mm)
    .. controls ++(9.5mm,-0.1mm) and ++(4.1mm,-3.9mm) .. ++(0.0mm,9.0mm)
    ;
    %% Group kvg:05b66-g4 --> top=False
    %% path id="kvg:05b66-s6"
    %% path spec="m 37.25,46.5 c 1,0.25 3.75,0.25 5.5,-0.25 1.75,-0.5 18.25,-4 20,-4 1.75,0 2.75,0.75 1,2.25 C 62,46 54.5,53.5 53,54.75"
    \draw (37.2mm,46.5mm)
    .. controls ++(1.0mm,0.2mm) and ++(-1.8mm,0.5mm) .. ++(5.5mm,-0.2mm)
    .. controls ++(1.8mm,-0.5mm) and ++(-1.8mm,0.0mm) .. ++(20.0mm,-4.0mm)
    .. controls ++(1.8mm,0.0mm) and ++(1.8mm,-1.5mm) .. ++(1.0mm,2.2mm)
    %%%% Warning: check controls
    .. controls (62.0mm,46.0mm) and (54.5mm,53.5mm) .. (53.0mm,54.8mm)
    ;
    %% path id="kvg:05b66-s7"
    %% path spec="m 50.75,55.75 c 4,8.75 7.18,24.67 1.75,38 -2.75,6.75 -7.75,1.25 -9.75,-2"
    \draw[/warp] (50.8mm,55.8mm)
    .. controls ++(4.0mm,8.8mm) and ++(5.4mm,-13.3mm) .. ++(1.8mm,38.0mm)
    .. controls ++(-2.8mm,6.8mm) and ++(2.0mm,3.2mm) .. ++(-9.8mm,-2.0mm)
    ;
    %% path id="kvg:05b66-s8"
    %% path spec="m 15.75,67.75 c 1.75,1 4.64,1.36 7.5,1 15.88,-2 44.43,-6.25 61.37,-5.5 2.5,0.11 4.72,0.25 6.39,1"
    \draw[/warp] (15.8mm,67.8mm)
    .. controls ++(1.8mm,1.0mm) and ++(-2.9mm,0.4mm) .. ++(7.5mm,1.0mm)
    .. controls ++(15.9mm,-2.0mm) and ++(-16.9mm,-0.8mm) .. ++(61.4mm,-5.5mm)
    .. controls ++(2.5mm,0.1mm) and ++(-1.7mm,-0.8mm) .. ++(6.4mm,1.0mm)
    ;
\end{scope}
        \end{scope}

		\fill[black]
			(0.5,6.5) node[right, scale=.7,name=inputlabel] {Input};
	\end{scope}

    \node (laba) [labelnode, left=5pt] at ($(inputlabel.west)$) {\textbf{(a)}};
    \node (labc) [labelnode] at ($(laba.mid)-(0,2)$) {\textbf{(c)}};
    \draw[tips, -{Latex[open,length=8pt]}] ($(laba.mid)+(15pt,-5pt)$) to [edge label'=\textbf{(b)}] ($(labc.mid)+(15pt,5pt)$);
    \node (labd) [labelnode, below left] at ($(d.mid)$) {\textbf{(d)}};
\begin{scope}[on background layer]
% Interlayer crossconnections
\draw (obstop) edge [thick, echoreg!65!black, arrows={-Latex[length=10pt, sep=-8pt]}, decoration={snake, segment length=2mm, amplitude=0.2mm}, decorate, double, in=90] (a);
\draw (obsbot) edge [thick, blue!70!black, arrows={-Latex[length=10pt, sep=-5pt]}, decoration={snake, segment length=2mm, amplitude=0.2mm}, double, decorate, out=90,in=90] (b);

\end{scope}

\end{tikzpicture}
        \caption[Summary of Proposed Network Architecture Flow]{The overall network flow, illustrating the intuition behind the use of bidirectional attention. Removed nodes and edges are indicated with dashed lines. Non-traversed edges are shown with thin lines.\newline
            \newline (a) The handwritten input character 学 (study) is given.
            \newline (b) The \textit{self-attention} of the Transformer first decides regions of focus for the \textit{bidirectional} cross attention.
            Forward attention is shown in orange-red, while the backwards attention is shown in blue.
            \newline (c) The first glimpse of the Observer, classifies both regions on the feature grid. Forward attention results in observation $\vec{f}_0$ = ($\tikz{\pic[scale=0.6] at (0, -5pt) {top};}$,\textbf{?}). Backwards attention results in observation $\vec{b}_0$ = ($\tikz{\pic[scale=0.6] at (0, -5pt) {bottom};}$,\textbf{子}).
            \newline (d) The bidirectional traversal removes edges from the opposing sparse trie with it's information.
                \newline\tab[0.5cm] - The position information of $\vec{f}_0$ and $\vec{b}_0$ removes connections to ⺾
                \newline\tab[1cm] because the non-compound bottom is present with a compound top, removing \textbf{荐} and \textbf{菰}.
                \newline\tab[0.5cm] - The remaining children of ⺾ are removed (\textbf{芥,苓,茶,荅,荼,莟,蒼,蓉,蔭,蓼,薈})
                \newline\tab[1cm] because the bottom constituent is not \textbf{子}.
                \newline\tab[0.5cm] - The $\tikz\pic[scale=0.6] at (0, -5pt) {top};$$\tikz\pic[color=black!70,scale=0.6] at (0pt, -5pt) {bottom};$ connection is then fully removed without needing to observe whether 冖 or 𠆢.
            \newline (e) The encoder-decoder pair determines the next area of attention unanimously at $\tikz\pic[scale=0.6] at (0, -5pt) {top};$$\tikz\pic[color=black!70,scale=0.6] at (0pt, -5pt) {top};$.
            \newline (f) The second observation sees \textbf{⺍}, and the extraneous information of 冖 is not needed to encode the input, saving an observation cycle. The differentiation between (学/study), (孛/character), and   (學/school) is achieved with the succinct encoding sequence:  $\tikz\pic[scale=0.6] at (0, -5pt) {bottom};$\textbf{子}$\tikz\pic[scale=0.6] at (0pt, -5pt) {top};$$\tikz\pic[color=black!70,scale=0.6] at (0pt, -5pt) {top};$\textbf{⺍}. When decoded, the result is \textbf{学}.
       }
        \label{fig:use}
    \end{center}
\end{figure*}
\newpage