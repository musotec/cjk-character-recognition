\definecolor{col1}{RGB}{255,0,0}
\definecolor{col2}{RGB}{0,255,0}
\definecolor{col3}{RGB}{0,0,255}

\newcommand{\SinCos}[4]{
    \begin{scope}[xscale=(5/#2)/2]
        \draw[color=#1!80,#3] (-1,0) sin (0,1) cos (1,0) sin (2,-1) cos (3,0) sin (4,1) cos (5,0) sin (6,-1) cos (7,0) sin (8,1) cos (9,0) sin (10,-1) cos (11,0) sin (12,1);
        \draw[color=#1!80!black,densely dash dot,xshift=28pt,#4] (-1,0) sin (0,1) cos (1,0) sin (2,-1) cos (3,0) sin (4,1) cos (5,0) sin (6,-1) cos (7,0) sin (8,1) cos (9,0) sin (10,-1) cos (11,0) sin (12,1);

        %        absolute value version
        %        \draw[color=#1!60] (-1,0) sin (0,1) cos (1,0) sin (2,1) cos (3,0) sin (4,1) cos (5,0) sin (6,1) cos (7,0) sin (8,1) cos (9,0) sin (10,1) cos (11,0) sin (12,1);
        %        \draw[color=#1!60!black, xshift=28pt] (-1,0) sin (0,1) cos (1,0) sin (2,1) cos (3,0) sin (4,1) cos (5,0) sin (6,1) cos (7,0) sin (8,1) cos (9,0) sin (10,1) cos (11,0) sin (12,1);
    \end{scope}
}

\begin{tikzpicture}[scale=0.9]
\begin{scope}[xshift=-150pt]
%    \begin{scope}[yshift=40pt] % Draw grid lines
%        \draw[help lines, color=red, dashed] (0,0) grid (5,5);
%        \draw[help lines, color=green, dashed] (0,0) grid[step=5/3] (5,5);
%        \draw[help lines, color=blue, dashed] (0,0) grid[step=5/2] (5,5);
%    \end{scope}
%

    \begin{scope}[rotate around={-90:(6.4,0)}, yscale=-1]
        \draw[<-] (-0.2,0) -- (5.2,0) node[below,name=y] {$y$};
        \draw[<->] (0,-1.3) -- (0,1.3);
        \draw (0,1) -- (-0.1,1) node[above] {1};
        \draw (0,-1) -- (-0.1,-1) node[above] {-1};
        \draw[<->] (5,-1.3) -- (5,1.3);
        \clip (0,-1.1) rectangle (5,1.1);
        \SinCos{red}{5}{}{ultra thick}
        \SinCos{green}{3}{}{ultra thick}
        \SinCos{blue}{2}{}{ultra thick}
    \end{scope}

    \begin{scope}
        \draw[<-] (-0.2,0) -- (5.2,0) node[right,name=x] {$x$};
        \draw[<->] (0,-1.3) -- (0,1.3);
        \draw (0,1) -- (-0.1,1) node[left] {1};
        \draw (0,-1) -- (-0.1,-1) node[left] {-1};
        \draw[<->] (5,-1.3) -- (5,1.3);

        \clip (0,-1.1) rectangle (5,1.1);

        \SinCos{red}{5}{}{ultra thick}
        \SinCos{green}{3}{}{ultra thick}
        \SinCos{blue}{2}{ultra thick}{}
    \end{scope}

    \draw (6.4,0) circle (3.5em) node[name=circ, text width=3em]{};
    \draw[thick] (6.4,0) circle (1.5em) node[align=center,below]{};
    \node at ($(circ.south)-(0,1.5cm)$) {\textit{positional\ encoding}};
    \draw[thick] ($(circ.center)$) to [controls=+(90:1em)and +(90:1em)] ($(circ.center)-(1.5em,0)$);
    \draw[thick] ($(circ.center)$) to [controls=+(-90:1em)and +(-90:1em)] ($(circ.center)+(1.5em,0)$);

    % draw kanji
    \begin{scope}[line cap=round, line width=3, scale=0.5, yscale=-1.05, yshift=-395, yslant=0.25]
    \pgfkeys{/warp/.style={color=black}}
    \useasboundingbox(0,0) rectangle (38.1mm,38.1mm);

    %% Group layer1 --> top=True
    %% Group kvg:StrokePaths_05b66 --> top=False
    %% Group kvg:05b66 --> top=False
    %% Group kvg:05b66-g1 --> top=False
    %% Group kvg:05b66-g2 --> top=False
    %% path id="kvg:05b66-s1"
    %% path spec="m 29.5,17.25 c 3.5,3 6.5,7.25 7.75,9.75"
    \draw[/warp] (29.5mm,17.2mm)
    .. controls ++(3.5mm,3.0mm) and ++(-1.2mm,-2.5mm) .. ++(7.8mm,9.8mm)
    ;
    %% path id="kvg:05b66-s2"
    %% path spec="m 49,12 c 1.25,2 4.75,8.25 5.25,11.5"
    \draw[/warp] (49.0mm,12.0mm)
    .. controls ++(1.2mm,2.0mm) and ++(-0.5mm,-3.2mm) .. ++(5.2mm,11.5mm)
    ;
    %% path id="kvg:05b66-s3"
    %% path spec="m 75,11 c 0.25,1.75 -0.12,2.75 -0.75,4.25 -1.29,3.1 -4.25,7.38 -6.5,9.75"
    \draw[/warp] (75.0mm,11.0mm)
    .. controls ++(0.2mm,1.8mm) and ++(0.6mm,-1.5mm) .. ++(-0.8mm,4.2mm)
    .. controls ++(-1.3mm,3.1mm) and ++(2.2mm,-2.4mm) .. ++(-6.5mm,9.8mm)
    ;
    %% Group kvg:05b66-g3 --> top=False
    %% path id="kvg:05b66-s4"
    %% path spec="M 21.25,33.75 C 21.13,38.5 19.25,46.25 17.5,50"
    \draw[/warp] (21.2mm,33.8mm)
    %%%% Warning: check controls
    .. controls (21.1mm,38.5mm) and (19.2mm,46.2mm) .. (17.5mm,50.0mm)
    ;
    %% path id="kvg:05b66-s5"
    %% path spec="m 23.5,36.5 c 17,-1.62 42.38,-5.5 60,-5.75 9.5,-0.13 4.12,5.12 0,9"
    \draw[/warp] (23.5mm,36.5mm)
    .. controls ++(17.0mm,-1.6mm) and ++(-17.6mm,0.2mm) .. ++(60.0mm,-5.8mm)
    .. controls ++(9.5mm,-0.1mm) and ++(4.1mm,-3.9mm) .. ++(0.0mm,9.0mm)
    ;
    %% Group kvg:05b66-g4 --> top=False
    %% path id="kvg:05b66-s6"
    %% path spec="m 37.25,46.5 c 1,0.25 3.75,0.25 5.5,-0.25 1.75,-0.5 18.25,-4 20,-4 1.75,0 2.75,0.75 1,2.25 C 62,46 54.5,53.5 53,54.75"
    \draw (37.2mm,46.5mm)
    .. controls ++(1.0mm,0.2mm) and ++(-1.8mm,0.5mm) .. ++(5.5mm,-0.2mm)
    .. controls ++(1.8mm,-0.5mm) and ++(-1.8mm,0.0mm) .. ++(20.0mm,-4.0mm)
    .. controls ++(1.8mm,0.0mm) and ++(1.8mm,-1.5mm) .. ++(1.0mm,2.2mm)
    %%%% Warning: check controls
    .. controls (62.0mm,46.0mm) and (54.5mm,53.5mm) .. (53.0mm,54.8mm)
    ;
    %% path id="kvg:05b66-s7"
    %% path spec="m 50.75,55.75 c 4,8.75 7.18,24.67 1.75,38 -2.75,6.75 -7.75,1.25 -9.75,-2"
    \draw[/warp] (50.8mm,55.8mm)
    .. controls ++(4.0mm,8.8mm) and ++(5.4mm,-13.3mm) .. ++(1.8mm,38.0mm)
    .. controls ++(-2.8mm,6.8mm) and ++(2.0mm,3.2mm) .. ++(-9.8mm,-2.0mm)
    ;
    %% path id="kvg:05b66-s8"
    %% path spec="m 15.75,67.75 c 1.75,1 4.64,1.36 7.5,1 15.88,-2 44.43,-6.25 61.37,-5.5 2.5,0.11 4.72,0.25 6.39,1"
    \draw[/warp] (15.8mm,67.8mm)
    .. controls ++(1.8mm,1.0mm) and ++(-2.9mm,0.4mm) .. ++(7.5mm,1.0mm)
    .. controls ++(15.9mm,-2.0mm) and ++(-16.9mm,-0.8mm) .. ++(61.4mm,-5.5mm)
    .. controls ++(2.5mm,0.1mm) and ++(-1.7mm,-0.8mm) .. ++(6.4mm,1.0mm)
    ;
\end{scope}

    % add selection grid
    \begin{scope}[transform shape, line cap=round,scale=0.5, yshift=80]

    %\draw[help lines] (10mm,10mm) grid (100mm,100mm);

    \def\wgrid{100mm}

    \draw[help lines, red, dashed, thick] (0mm,0mm) grid [step=\wgrid/5] (\wgrid,\wgrid);
    \draw[help lines, green!80!black, dashed, thick] (0mm,0mm) grid [step=\wgrid/3] (\wgrid,\wgrid);
    \draw[help lines, blue, dashed, thick] (0mm,0mm) grid [step=\wgrid/2] (\wgrid,\wgrid);
    \clip (0mm, 0mm) rectangle (\wgrid, \wgrid);

    \def\gx{1}
    \def\gy{2}
    \def\gxx{5}
    \def\bx{1}
    \def\by{0}
    \def\rx{1}
    \def\ry{1}
    \def\rwidth{3}
    \newcommand{\dr}{\wgrid/10}
    \newcommand{\dg}{\wgrid/6}
    \newcommand{\db}{\wgrid/4}

    \newcommand{\AttentionGrid}[2]{
        \node (blue) [rounded corners=1pt, #1=blue!90!black, minimum height=2*\db, minimum width=2*\db, #2] at (\db+\bx*\db,\db+\by*\db) {};
        \node (red) [rounded corners=1pt, #1=red!90!black, opacity=0.8, minimum height=2*\rwidth*\dr, minimum width=2*\rwidth*\dr, #2] at (\dr + \rx*\dr + \rx*\rwidth*\dr,\dr + \ry*\rwidth*\dr + \ry*\dr) {};
        \node (green) [rounded corners=1pt, #1=green!80!black, minimum height=2*\dg, minimum width=2*\dg, #2] at (\gx*\dg,\dg+\gy*\dg) {};

        \node (green2) [rounded corners=1pt, #1=green!80!black, minimum height=2*\dg, minimum width=2*\dg, #2] at (\gxx*\dg,\dg+\gy*\dg) {};
    }

    \begin{scope}[transparency group, fill opacity=0.55, draw opacity=1]
        \begin{scope}[blend group=screen]
            \AttentionGrid{fill}{}
        \end{scope}
    \end{scope}
    %
    \begin{scope}[transparency group, opacity=0.8, every node/.style={fill=none}]
        \AttentionGrid{ultra thick,draw,color}{remember picture}
    \end{scope}
\end{scope}

\end{scope}

\begin{scope}[yshift=0,xshift=0+175,on background layer]
	%% Vanishing points for perspective handling
	\coordinate (P1) at (-8cm,1.5cm); % left vanishing point (To pick)
	\coordinate (P2) at (8cm,1.5cm);  % right vanishing point (To pick)

	%% (A1) and (A2) defines the 2 central points of the cuboid
	\coordinate (A1) at (0em,0cm); % central top point (To pick)
	\coordinate (A2) at (0em,-2cm); % central bottom point (To pick)

    \coordinate (X1) at ($(P2)!.5!(A1)$);
    \coordinate (X2) at ($(P2)!.5!(A2)$);

    % TODO: can extract calculation w/ perspective to get perceived length equal to A2 (-2cm)
    % TODO: equation solutions to floats here:
    %    above = y_dist(a1 - p1); below = y_dist(a2 - p1)
    %    phi = atan(y/x)
    %    s*90 = 180 - 2*phi; solve for s
    %    x` = s * (above/below)
    %    y` = s * sin(phi)

    %% (A3) to (A8) are computed given a unique parameter (or 2) .8
	% You can vary .8 from 0 to 1 to change perspective on left side
	\coordinate (A3) at ($(P1)!.8346!(A2)$); % To pick for perspective
	\coordinate (A4) at ($(P1)!.8346!(A1)$);

	% You can vary .8 from 0 to 1 to change perspective on right side
	\coordinate (A7) at ($(P2)!.9375!(A2)$);
	\coordinate (A8) at ($(P2)!.9375!(A1)$);

	%% Automatically compute the last 2 points with intersections
	\coordinate (A5) at
	  (intersection cs: first line={(A8) -- (P1)},
			    second line={(A4) -- (P2)});
	\coordinate (A6) at
	  (intersection cs: first line={(A7) -- (P1)},
			    second line={(A3) -- (P2)});


    \coordinate (B1) at ($(P2)!.995!(A8)$);
    \coordinate (B2) at ($(P2)!.995!(A7)$);

    \coordinate (B3) at ($(P2)!.995!(A6)$);
    \coordinate (B4) at ($(P2)!.995!(A5)$);

    \coordinate (B7) at ($(P2)!.775!(A2)$);
    \coordinate (B8) at ($(P2)!.775!(A1)$);


    \coordinate (B5) at
      (intersection cs: first line={(B8) -- (P1)},
                second line={(A5) -- (P2)});

    \coordinate (B6) at
      (intersection cs: first line={(B7) -- (P1)},
                second line={(A6) -- (P2)});



    \coordinate (C1) at ($(P2)!.995!(B8)$);
    \coordinate (C2) at ($(P2)!.995!(B7)$);

    \coordinate (C3) at ($(P2)!.995!(B6)$);
    \coordinate (C4) at ($(P2)!.995!(B5)$);

    \coordinate (C7) at ($(P2)!.625!(B2)$);
    \coordinate (C8) at ($(P2)!.625!(B1)$);


    \coordinate (C5) at
    (intersection cs: first line={(C8) -- (P1)},
    second line={(B5) -- (P2)});

    \coordinate (C6) at
    (intersection cs: first line={(C7) -- (P1)},
    second line={(B6) -- (P2)});

	%%%%%%%%%%%%%%%%%%%%%%
    %%%%%%%%%%%%%%%%%%%%%% Start filling

    \fill[col1!70!black] (C2) -- (C3) -- (C6) -- (C7) -- cycle; % bottom face
    \fill[col1!50] (C3) -- (C4) -- (C5) -- (C6) -- cycle;  % far left face
    \fill[col1!30,opacity=0.5] (C5) -- (C6) -- (C7) -- (C8) -- cycle; % far right face
	\draw[thick,dashed] (C5) -- (C6);
    \draw[thick,dashed] (C3) -- (C6);
    \draw[thick,dashed] (C7) -- (C6);
    \fill[col1,opacity=0.7] (C1) -- (C8) -- (C7) -- (C2) -- cycle; % face 1
    \node at (barycentric cs:C1=1,C8=1,C7=1,C2=1) {}; % \tiny c1};
    \fill[col1,opacity=0.3] (C1) -- (C2) -- (C3) -- (C4) -- cycle; % c2
    \fill[col1,opacity=0.7] (C1) -- (C4) -- (C5) -- (C8) -- cycle; % c5
    \node (t1) [anchor=north] at (barycentric cs:C1=1,C4=1,C5=1,C8=1) {};
    \draw[thick] (C1) -- (C2);
    \draw[thick] (C3) -- (C4);
    \draw[thick] (C7) -- (C8);
    \draw[thick] (C1) -- (C4);
    \draw[thick] (C1) -- (C8);
    \draw[thick] (C2) -- (C3);
    \draw[thick] (C2) -- (C7);
    \draw[thick] (C4) -- (C5);
    \draw[thick] (C8) -- (C5);

    %\fill[green!70!black] (B2) -- (B3) -- (B6) -- (B7) -- cycle;
    %\fill[green!50!black] (B3) -- (B4) -- (B5) -- (B6) -- cycle;
    %\fill[green!30] (B5) -- (B6) -- (B7) -- (B8) -- cycle; % far face

    \fill[col2!70!black] (B2) -- (B3) -- (B6) -- (B7) -- cycle;
    \fill[col2!50] (B3) -- (B4) -- (B5) -- (B6) -- cycle;
    \fill[col2!30,opacity=0.5] (B5) -- (B6) -- (B7) -- (B8) -- cycle; % far right face

    \draw[thick,dashed] (B5) -- (B6);
    \draw[thick,dashed] (B3) -- (B6);
    \draw[thick,dashed] (B7) -- (B6);

    \fill[col2,opacity=0.7] (B1) -- (B8) -- (B7) -- (B2) -- cycle; % face 1
    \node at (barycentric cs:B1=1,B8=1,B7=1,B2=1) {}; % \tiny b1};

    \fill[col2,opacity=0.3] (B1) -- (B2) -- (B3) -- (B4) -- cycle; % b2
    %\node at (barycentric cs:B1=1,B2=1,B3=1,B4=1) {\tiny b2};

    \fill[col2,opacity=0.7] (B1) -- (B4) -- (B5) -- (B8) -- cycle; % f5
    \node (t2) [anchor=north] at (barycentric cs:B1=1,B4=1,B5=1,B8=1) {};


    \draw[thick] (B1) -- (B2);
    \draw[thick] (B3) -- (B4);
    \draw[thick] (B7) -- (B8);
    \draw[thick] (B1) -- (B4);
    \draw[thick] (B1) -- (B8);
    \draw[thick] (B2) -- (B3);
    \draw[thick] (B2) -- (B7);
    \draw[thick] (B4) -- (B5);
    \draw[thick] (B8) -- (B5);

    %%%
    %%%

    %% Possibly draw back faces
    \fill[col3!70!black,opacity=0.7] (A2) -- (A3) -- (A6) -- (A7) -- cycle; % face 6
    %\node at (barycentric cs:A2=1,A3=1,A6=1,A7=1) {\tiny f6};

    \fill[col3!50!black,opacity=0.7] (A3) -- (A4) -- (A5) -- (A6) -- cycle; % face 3
    %\node at (barycentric cs:A3=1,A4=1,A5=1,A6=1) {\tiny f3};

    \fill[col3!30,opacity=0.5] (A5) -- (A6) -- (A7) -- (A8) -- cycle; % face 4
    %\node at (barycentric cs:A5=1,A6=1,A7=1,A8=1) {\tiny f4};

	\draw[thick,dashed] (A5) -- (A6);
	\draw[thick,dashed] (A3) -- (A6);
	\draw[thick,dashed] (A7) -- (A6);


	%% Possibly draw front faces

	\fill[col3,opacity=0.7] (A1) -- (A8) -- (A7) -- (A2) -- cycle; % face 1

	\fill[col3!70!black,opacity=0.7] (A1) -- (A2) -- (A3) -- (A4) -- cycle; % f2

	\fill[col3,opacity=0.7] (A1) -- (A4) -- (A5) -- (A8) -- cycle; % f5

    \node at (barycentric cs:A1=1,A8=1,A7=1,A2=1) {};%\tiny a1};

    \node at (barycentric cs:A1=1,A2=1,A3=1,A4=1) {};%\tiny a2};

    \node (t3) [anchor=north] at (barycentric cs:A1=1,A4=1,A5=1,A8=1) {};

	%% Possibly draw front lines
	\draw[thick] (A1) -- (A2);
	\draw[thick] (A3) -- (A4);
	\draw[thick] (A7) -- (A8);
	\draw[thick] (A1) -- (A4);
	\draw[thick] (A1) -- (A8);
	\draw[thick] (A2) -- (A3);
	\draw[thick] (A2) -- (A7);
	\draw[thick] (A4) -- (A5);
	\draw[thick] (A8) -- (A5);

    % Draw debug points
%	\foreach \i in {1,2,...,8}
%	{
%	  \draw[fill=black] (A\i) circle (0.25em)
%	    node[color=white] {\tiny \i};
%	}

    %\draw[fill=black] (P1) circle (0.1em) node[below] {\tiny p1};
	%\draw[fill=black] (P2) circle (0.1em) node[below] {\tiny p2};

\end{scope}

\begin{scope}[yshift=64,xshift=260,z={(10:10mm)},x={(165:10mm)}]
    \begin{scope}[canvas is zx plane at y=0, scale=1.125]

    % Draw the Kanji stroke on the grid
    \begin{scope}[line cap=round, line width=3, scale=0.5, yscale=-1.05, yshift=-395, yslant=0.25]
    \pgfkeys{/warp/.style={color=black}}
    \useasboundingbox(0,0) rectangle (38.1mm,38.1mm);

    %% Group layer1 --> top=True
    %% Group kvg:StrokePaths_05b66 --> top=False
    %% Group kvg:05b66 --> top=False
    %% Group kvg:05b66-g1 --> top=False
    %% Group kvg:05b66-g2 --> top=False
    %% path id="kvg:05b66-s1"
    %% path spec="m 29.5,17.25 c 3.5,3 6.5,7.25 7.75,9.75"
    \draw[/warp] (29.5mm,17.2mm)
    .. controls ++(3.5mm,3.0mm) and ++(-1.2mm,-2.5mm) .. ++(7.8mm,9.8mm)
    ;
    %% path id="kvg:05b66-s2"
    %% path spec="m 49,12 c 1.25,2 4.75,8.25 5.25,11.5"
    \draw[/warp] (49.0mm,12.0mm)
    .. controls ++(1.2mm,2.0mm) and ++(-0.5mm,-3.2mm) .. ++(5.2mm,11.5mm)
    ;
    %% path id="kvg:05b66-s3"
    %% path spec="m 75,11 c 0.25,1.75 -0.12,2.75 -0.75,4.25 -1.29,3.1 -4.25,7.38 -6.5,9.75"
    \draw[/warp] (75.0mm,11.0mm)
    .. controls ++(0.2mm,1.8mm) and ++(0.6mm,-1.5mm) .. ++(-0.8mm,4.2mm)
    .. controls ++(-1.3mm,3.1mm) and ++(2.2mm,-2.4mm) .. ++(-6.5mm,9.8mm)
    ;
    %% Group kvg:05b66-g3 --> top=False
    %% path id="kvg:05b66-s4"
    %% path spec="M 21.25,33.75 C 21.13,38.5 19.25,46.25 17.5,50"
    \draw[/warp] (21.2mm,33.8mm)
    %%%% Warning: check controls
    .. controls (21.1mm,38.5mm) and (19.2mm,46.2mm) .. (17.5mm,50.0mm)
    ;
    %% path id="kvg:05b66-s5"
    %% path spec="m 23.5,36.5 c 17,-1.62 42.38,-5.5 60,-5.75 9.5,-0.13 4.12,5.12 0,9"
    \draw[/warp] (23.5mm,36.5mm)
    .. controls ++(17.0mm,-1.6mm) and ++(-17.6mm,0.2mm) .. ++(60.0mm,-5.8mm)
    .. controls ++(9.5mm,-0.1mm) and ++(4.1mm,-3.9mm) .. ++(0.0mm,9.0mm)
    ;
    %% Group kvg:05b66-g4 --> top=False
    %% path id="kvg:05b66-s6"
    %% path spec="m 37.25,46.5 c 1,0.25 3.75,0.25 5.5,-0.25 1.75,-0.5 18.25,-4 20,-4 1.75,0 2.75,0.75 1,2.25 C 62,46 54.5,53.5 53,54.75"
    \draw (37.2mm,46.5mm)
    .. controls ++(1.0mm,0.2mm) and ++(-1.8mm,0.5mm) .. ++(5.5mm,-0.2mm)
    .. controls ++(1.8mm,-0.5mm) and ++(-1.8mm,0.0mm) .. ++(20.0mm,-4.0mm)
    .. controls ++(1.8mm,0.0mm) and ++(1.8mm,-1.5mm) .. ++(1.0mm,2.2mm)
    %%%% Warning: check controls
    .. controls (62.0mm,46.0mm) and (54.5mm,53.5mm) .. (53.0mm,54.8mm)
    ;
    %% path id="kvg:05b66-s7"
    %% path spec="m 50.75,55.75 c 4,8.75 7.18,24.67 1.75,38 -2.75,6.75 -7.75,1.25 -9.75,-2"
    \draw[/warp] (50.8mm,55.8mm)
    .. controls ++(4.0mm,8.8mm) and ++(5.4mm,-13.3mm) .. ++(1.8mm,38.0mm)
    .. controls ++(-2.8mm,6.8mm) and ++(2.0mm,3.2mm) .. ++(-9.8mm,-2.0mm)
    ;
    %% path id="kvg:05b66-s8"
    %% path spec="m 15.75,67.75 c 1.75,1 4.64,1.36 7.5,1 15.88,-2 44.43,-6.25 61.37,-5.5 2.5,0.11 4.72,0.25 6.39,1"
    \draw[/warp] (15.8mm,67.8mm)
    .. controls ++(1.8mm,1.0mm) and ++(-2.9mm,0.4mm) .. ++(7.5mm,1.0mm)
    .. controls ++(15.9mm,-2.0mm) and ++(-16.9mm,-0.8mm) .. ++(61.4mm,-5.5mm)
    .. controls ++(2.5mm,0.1mm) and ++(-1.7mm,-0.8mm) .. ++(6.4mm,1.0mm)
    ;
\end{scope}

    \begin{scope}[transform shape, line cap=round,scale=0.5, yshift=80]

    %\draw[help lines] (10mm,10mm) grid (100mm,100mm);

    \def\wgrid{100mm}

    \draw[help lines, red, dashed, thick] (0mm,0mm) grid [step=\wgrid/5] (\wgrid,\wgrid);
    \draw[help lines, green!80!black, dashed, thick] (0mm,0mm) grid [step=\wgrid/3] (\wgrid,\wgrid);
    \draw[help lines, blue, dashed, thick] (0mm,0mm) grid [step=\wgrid/2] (\wgrid,\wgrid);
    \clip (0mm, 0mm) rectangle (\wgrid, \wgrid);

    \def\gx{1}
    \def\gy{2}
    \def\gxx{5}
    \def\bx{1}
    \def\by{0}
    \def\rx{1}
    \def\ry{1}
    \def\rwidth{3}
    \newcommand{\dr}{\wgrid/10}
    \newcommand{\dg}{\wgrid/6}
    \newcommand{\db}{\wgrid/4}

    \newcommand{\AttentionGrid}[2]{
        \node (blue) [rounded corners=1pt, #1=blue!90!black, minimum height=2*\db, minimum width=2*\db, #2] at (\db+\bx*\db,\db+\by*\db) {};
        \node (red) [rounded corners=1pt, #1=red!90!black, opacity=0.8, minimum height=2*\rwidth*\dr, minimum width=2*\rwidth*\dr, #2] at (\dr + \rx*\dr + \rx*\rwidth*\dr,\dr + \ry*\rwidth*\dr + \ry*\dr) {};
        \node (green) [rounded corners=1pt, #1=green!80!black, minimum height=2*\dg, minimum width=2*\dg, #2] at (\gx*\dg,\dg+\gy*\dg) {};

        \node (green2) [rounded corners=1pt, #1=green!80!black, minimum height=2*\dg, minimum width=2*\dg, #2] at (\gxx*\dg,\dg+\gy*\dg) {};
    }

    \begin{scope}[transparency group, fill opacity=0.55, draw opacity=1]
        \begin{scope}[blend group=screen]
            \AttentionGrid{fill}{}
        \end{scope}
    \end{scope}
    %
    \begin{scope}[transparency group, opacity=0.8, every node/.style={fill=none}]
        \AttentionGrid{ultra thick,draw,color}{remember picture}
    \end{scope}
\end{scope}
    \end{scope}

    \begin{scope}[canvas is zy plane at x=0]
        \node (head) at ($(blue.south)+(0.25,-0.25)$) [transform shape] {multi-head attention};
    \end{scope}

    \begin{scope}[z={(12.5:7mm)},canvas is zy plane at x=0]
        \node (conv) at ($(blue.south)+(0,-1.1)$) [transform shape] {{\footnotesize convolutional observation network}};
    \end{scope}

    \begin{scope}[z={(15:5mm)},canvas is zy plane at x=0]
        \draw[transform shape, ->] (head) -> (conv);
        \node (conc) at ($(blue.south)+(-1.5,-2)$) [transform shape] {concatenate};
    \end{scope}

\end{scope}
\begin{scope}[xshift=260,z={(26:10mm)},x={(165:15mm)}] %,y={(-155:10mm)}]
    \begin{scope}[canvas is zx plane at y=-200]
        \node at ($(B7.south)+(-0.5, -0.5)$) [transform shape] {encoder-decoder attention};
    \end{scope}
\end{scope}

\draw[->] (conv) -> (conc);

\begin{scope}[on background layer, out=-90,in=90,looseness=1.5, opacity=0.7]
    \draw[red!80!black, ultra thick, double distance=1pt] (red.mid) to (t1);
    \draw[green!80!black,ultra thick, double distance=1pt] (green.mid) to (t2);
    \draw[green!80!black,ultra thick, double distance=1pt] (green2.mid) to (t2);
    \draw[blue!80!black,ultra thick, double distance=1pt] (blue.mid) to (t3);
\end{scope}

\end{tikzpicture}
