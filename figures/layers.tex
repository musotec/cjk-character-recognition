%\documentclass[crop, tikz]{standalone}
%\usepackage{tikz}
%\usepackage{amsmath}
%
%\usetikzlibrary{decorations.pathmorphing, positioning, backgrounds, arrows.meta, calc}

\definecolor{echoreg}{HTML}{ff4400}
\definecolor{echodrk}{HTML}{0044ff}

\tikzstyle{mynode} = [text=black, very thick,
    rectangle, inner sep=10pt, inner ysep=20pt]
\tikzstyle{fancytitle} =[text=black]


\tikzstyle{labelnode} = [text=black, very thick,
draw=none, font=\normalsize, outer sep=0pt]

\newcommand{\yslant}{0.5}
\newcommand{\xslant}{-0.6}

\begin{tikzpicture}[scale=0.58,every node/.style={minimum size=1cm},on grid]

	\node [mynode, scale=0.85] at (10.5, 0) (box){%
		\begin{minipage}{0.2\textwidth}

          \tikz[
              yshift=50,
              edge from parent path={(\tikzparentnode.south) -- (\tikzchildnode.north)},
              every child node/.style={color=black,scale=0.35,thick,outer sep=0pt,circle,minimum size=16pt,draw,column sep={1.5pt,between origins}},
              font=\Huge
          ] {

          % define the set of forward observed nodes
          \node (fA) [draw, thick, circle, scale=0.35, fill=echoreg!20] at (0,0) {A}; % 十 pin={0:\scriptsize{t}}
          \node (fB) [draw, thick, circle, scale=0.35, fill=echoreg!20] at (0,-1) {\textbf{?}};
          \node (fX) [draw, thin, circle, scale=0.35, fill=white] at (1,-1) {宀}; % 宀 (字)
          \pic [scale=0.9] at ($(fA.mid)-(0.225,0.155)$) {top};
          \node (fC) [draw, thick, circle, scale=0.35, fill=echoreg!40, double] at (0,-2) {C};
          \pic [scale=0.9] at ($(fC.mid)-(0.225,0.155)$) {top};
          \node (fD) [draw, dashed, thick, circle, scale=0.35, fill=white] at (1,-2) {D};
          \pic [scale=0.9] at ($(fD.mid)-(0.225,0.155)$) {bottom};

          % define the set of result nodes
          \begin{scope}[xshift=5,yshift=5,scale=1.1,
              every pin edge/.style={draw=none}]
              \node (rA) [draw, thick, circle, scale=0.35, fill=white, dashed] at (0,-4) {⺾}; % ⺾ (芥) 1st CUT [荐,菰 because bottom is compound, top is not]; also cuts 芥苓茶荅荼莟蒼蓉蔭蓼薈... after first observation without having to look at [top][bottom] to determine if it is 冖 or 𠆢. ⺾/𠆢 in this path is fully cut before observation of [top][top] because ALL possible bottom characters in the tree are not 子.

              \node (rB) [draw, thick, circle, scale=0.35, fill=white, pin={[pin distance=-15pt]below:\textsf{孛}}] at (1,-4) {十}; % 十 (孛/character)(學/school)
              \node (rC) [draw, thick, circle, scale=0.35, fill=echoreg!50!echodrk!50, double, pin={[name=p1,pin distance=-15pt]below:\textbf{\underline{学}}}] at (2,-4) {\textbf{⺍}}; % ⺍ (学/study)
              \node (rD) [draw, thick, circle, scale=0.35, fill=white, pin={[pin distance=-15pt]below:學}] at (3,-4) {𦥯}; % 𦥯 (學/school)
              \node[labelnode, above=-15pt, scale=1.15] at ($(p1.south)$) {\textbf{(f)}};
          \end{scope}
          % 學学孛字孕

%          \node (D) [draw, thick, circle, scale=0.35, fill=echoreg!40] at (2,-5) {D};

%  電子 (でんしゃ) means "electron" but literally translates to "electric child", 孛 combines "above" and "child"



          \begin{scope}[xshift=70]
              \node (bA) [draw, thick, circle, scale=0.35, fill=echodrk!20] at (0,0) {A};
              \node (bB) [draw, thick, circle, scale=0.35, fill=echodrk!20] at (0,-1) {\textbf{子}};
              \pic [scale=0.9] at ($(bA.mid)-(0.225,0.155)$) {bottom};
              \node (bC) [draw, thick, circle, scale=0.35, fill=echodrk!40] at (1,-1) {C};
              \pic [scale=0.9] at ($(bC.mid)-(0.225,0.155)$) {top};

              \node (bD) [draw, thick, circle, scale=0.35, fill=echodrk!40, double] at (0,-2) {D};
              \pic [scale=0.9] at ($(bD.mid)-(0.225,0.155)$) {top};

              \node (bE) [draw, thick, circle, scale=0.35, fill=white, dashed] at (1,-2) {E};
              \pic [scale=0.9] at ($(bE.mid)-(0.225,0.155)$) {bottom};
          \end{scope}

          \node[labelnode, scale=1.15] at ($(fD)!.5!(bD) - (0,15pt)$) {\textbf{(e)}};
          \draw[ultra thick] (fA) -- (fB);
          \draw[ultra thick] (fB) -- (fC);
          \draw[thin] (fA) -- (fX);
          \draw[densely dashed] (fB) -- (fD);

          \draw[ultra thick] (bA) -- (bB);
          \draw[ultra thick] (bB) -- (bC);
          \draw[ultra thick] (bC) -- (bD);
          \draw[densely dashed] (bC) -- (bE);

          \draw[densely dashed] (fC) -- (rA);
          \draw (fC) -- (rB);
          \draw[ultra thick] (fC) -- (rC);
          \draw (fC) -- (rD);

          \draw (bD) -- (rB);
          \draw[ultra thick] (bD) -- (rC);
          \draw (bD) -- (rD);

          }
    	\end{minipage}
	};

	\node[fancytitle, scale=0.8] at (box.north) {\bf Sparse Trie Traversal:};

	% Layer 2
	\begin{scope}[
		yshift=-250,
		every node/.append style={yslant=\yslant,xslant=\xslant},
		yslant=\yslant,xslant=\xslant,
        on background layer
	]
        \draw[black, dashed, thin] (0,0) rectangle (7,7);

        \begin{scope}[yshift=-20,xshift=40]
         \begin{scope}[yshift=40]
            \def\wgrid{50mm}
            \def\wbox{\wgrid/4}

            \draw[help lines, black!40] (0mm,0mm) grid [step=\wgrid/4] (\wgrid, \wgrid);

            % Draw nodes and labels on graph
            \begin{scope}[xshift=0.5*\wbox, yshift=0.5*\wbox]
                % nodes
                \node (a) [draw, thick, circle, scale=0.5, fill=echoreg] at (0*\wbox, 3*\wbox) {};
                \draw[fill=echodrk] (3*\wbox,0*\wbox) circle (.1);
                \node (b) [draw, thick, circle, scale=0.5, fill=echodrk] at (3*\wbox, 0*\wbox) {};
                \node (c) [draw, double, circle, scale=0.5, fill=echodrk!50!echoreg] at (2*\wbox, 2*\wbox) {};
                \node (d) [draw, densely dashed, circle, scale=0.5] at (1*\wbox, 1*\wbox) {};

                % top labels
                \pic[scale=0.45, color=black, anchor=south] at (3*\wbox-10,4*\wbox-10) {bottom};
                \node [scale=0.75,anchor=west] at (3*\wbox-10,4*\wbox-10) {\textbf{子}};
                \pic[scale=0.45, color=black, anchor=south] at (0*\wbox-10,4*\wbox-10) {top};
                \node [scale=0.75,anchor=west] at (0*\wbox-10,4*\wbox-10) {\textbf{?}};
                \pic[scale=0.45, color=black, anchor=south] at (1*\wbox-10,4*\wbox+3) {top};
                \pic[scale=0.45, color=black!80, anchor=south] at (1*\wbox-10,4*\wbox-13) {bottom};
                \node [scale=0.75,anchor=west] at (1*\wbox-10,4*\wbox-10) {\textbf{冖}};
                \pic[scale=0.45, color=black, anchor=south] at (2*\wbox-10,4*\wbox+3) {top};
                \pic[scale=0.45, color=black!80, anchor=south] at (2*\wbox-10,4*\wbox-13) {top};
                \node [scale=0.75,anchor=west] at (2*\wbox-10,4*\wbox-10) {\textbf{⺍}};

                % left labels
                \pic[scale=0.45, color=black, anchor=south] at (-1*\wbox,0*\wbox-10) {bottom};
                \pic[scale=0.45, color=black, anchor=south] at (-1*\wbox,3*\wbox-10) {top};
                \pic[scale=0.45, color=black, anchor=south] at (-1*\wbox,1*\wbox+3) {top};
                \pic[scale=0.45, color=black!80, anchor=south] at (-1*\wbox,1*\wbox-13) {bottom};
                \pic[scale=0.45, color=black, anchor=south] at (-1*\wbox,2*\wbox+3) {top};
                \pic[scale=0.45, color=black!80, anchor=south] at (-1*\wbox,2*\wbox-13) {top};
            \end{scope}
         \end{scope}
        \end{scope}

        % Draw edges between nodes
        \draw[-latex, thick, color=echoreg] (a) -- (c);
		\draw[-latex, thick, color=echodrk] (b) -- (c);
        \draw (a) edge [draw, -latex, thin, dashed, color=black!70] (d);
        \draw (c) edge [draw, -latex, thin, dashed, bend right=35, looseness=1.25, color=echoreg] (d);
        \draw (c) edge [draw, -latex, thin, dashed, bend right=-35, looseness=1.25, color=echodrk] (d);

        \fill[black] (1.75,0.35) node[right, scale=.7] {Bidirectional Attention};
	\end{scope}

    % Observation Layer
    \begin{scope}[
        yshift=-60,
        every node/.append style={yslant=\yslant,xslant=\xslant},
        yslant=\yslant,xslant=\xslant
        ]
        \fill[white,fill opacity=.75] (0,0) rectangle (7,7);
        \draw[black, dashed, thin] (0,0) rectangle (7,7);

        \begin{scope}[yshift=-20,xshift=20]
            \begin{scope}[yshift=40]
                \def\wgrid{50mm}
                \def\wbox{\wgrid/16}

                \draw[help lines, black!40] (0mm,0mm) grid [step=\wgrid/16] (\wgrid, \wgrid);
                % Create the pixel labelings for the observations
                \begin{scope}[transparency group, opacity=0.6]
                    \fill[fill=black!30] (0,0) rectangle (\wgrid, \wgrid);

                    \fill[fill=echoreg] (\wbox*2, \wbox*9) rectangle (\wbox*4, \wbox*13);
                    \fill[fill=echoreg] (\wbox*4, \wbox*15) rectangle (\wbox*13, \wbox*10);
                    \fill[fill=echoreg] (\wbox*12, \wbox*11) rectangle (\wbox*15, \wbox*8);

                    \fill[fill=blue] (\wbox*10, \wbox*7) rectangle (\wbox*15, \wbox*3);
                    \fill[fill=blue] (\wbox*5, \wbox*10) rectangle (\wbox*11, \wbox*8);
                    \fill[fill=blue] (\wbox*6, \wbox*10) rectangle (\wbox*11, \wbox*0);
                    \fill[fill=blue] (\wbox*2, \wbox*8) rectangle (\wbox*8, \wbox*4);

                    \node (obstop) [draw=none] at (\wgrid/2,\wbox*13) {};
                    \node (obsbot) [draw=none] at (\wgrid/2,\wbox*6) {};
                \end{scope}
            \end{scope}
        \end{scope}

        \fill[black] (3.75,0.35) node[right, scale=.7] {Observations};
    \end{scope}

	% Input Layer
	\begin{scope}[
		yshift=0,
		every node/.append style={yslant=\yslant,xslant=\xslant},
		yslant=\yslant,xslant=\xslant
	]
		\fill[white,fill opacity=.75] (0,0) rectangle (7,7);
		\draw[black, dashed, thin] (0,0) rectangle (7,7);

        \begin{scope}[yshift=-20,xshift=20]
            \begin{scope}[yshift=40]
                \def\wgrid{50mm}
                \def\wbox{\wgrid/16}

                \draw[help lines, black!40] (0mm,0mm) grid [step=\wbox] (\wgrid, \wgrid);

            \end{scope}

            \begin{scope}[line cap=round, line width=3, scale=0.5, yscale=-1.05, yshift=-395, yslant=0.25]
    \pgfkeys{/warp/.style={color=black}}
    \useasboundingbox(0,0) rectangle (38.1mm,38.1mm);

    %% Group layer1 --> top=True
    %% Group kvg:StrokePaths_05b66 --> top=False
    %% Group kvg:05b66 --> top=False
    %% Group kvg:05b66-g1 --> top=False
    %% Group kvg:05b66-g2 --> top=False
    %% path id="kvg:05b66-s1"
    %% path spec="m 29.5,17.25 c 3.5,3 6.5,7.25 7.75,9.75"
    \draw[/warp] (29.5mm,17.2mm)
    .. controls ++(3.5mm,3.0mm) and ++(-1.2mm,-2.5mm) .. ++(7.8mm,9.8mm)
    ;
    %% path id="kvg:05b66-s2"
    %% path spec="m 49,12 c 1.25,2 4.75,8.25 5.25,11.5"
    \draw[/warp] (49.0mm,12.0mm)
    .. controls ++(1.2mm,2.0mm) and ++(-0.5mm,-3.2mm) .. ++(5.2mm,11.5mm)
    ;
    %% path id="kvg:05b66-s3"
    %% path spec="m 75,11 c 0.25,1.75 -0.12,2.75 -0.75,4.25 -1.29,3.1 -4.25,7.38 -6.5,9.75"
    \draw[/warp] (75.0mm,11.0mm)
    .. controls ++(0.2mm,1.8mm) and ++(0.6mm,-1.5mm) .. ++(-0.8mm,4.2mm)
    .. controls ++(-1.3mm,3.1mm) and ++(2.2mm,-2.4mm) .. ++(-6.5mm,9.8mm)
    ;
    %% Group kvg:05b66-g3 --> top=False
    %% path id="kvg:05b66-s4"
    %% path spec="M 21.25,33.75 C 21.13,38.5 19.25,46.25 17.5,50"
    \draw[/warp] (21.2mm,33.8mm)
    %%%% Warning: check controls
    .. controls (21.1mm,38.5mm) and (19.2mm,46.2mm) .. (17.5mm,50.0mm)
    ;
    %% path id="kvg:05b66-s5"
    %% path spec="m 23.5,36.5 c 17,-1.62 42.38,-5.5 60,-5.75 9.5,-0.13 4.12,5.12 0,9"
    \draw[/warp] (23.5mm,36.5mm)
    .. controls ++(17.0mm,-1.6mm) and ++(-17.6mm,0.2mm) .. ++(60.0mm,-5.8mm)
    .. controls ++(9.5mm,-0.1mm) and ++(4.1mm,-3.9mm) .. ++(0.0mm,9.0mm)
    ;
    %% Group kvg:05b66-g4 --> top=False
    %% path id="kvg:05b66-s6"
    %% path spec="m 37.25,46.5 c 1,0.25 3.75,0.25 5.5,-0.25 1.75,-0.5 18.25,-4 20,-4 1.75,0 2.75,0.75 1,2.25 C 62,46 54.5,53.5 53,54.75"
    \draw (37.2mm,46.5mm)
    .. controls ++(1.0mm,0.2mm) and ++(-1.8mm,0.5mm) .. ++(5.5mm,-0.2mm)
    .. controls ++(1.8mm,-0.5mm) and ++(-1.8mm,0.0mm) .. ++(20.0mm,-4.0mm)
    .. controls ++(1.8mm,0.0mm) and ++(1.8mm,-1.5mm) .. ++(1.0mm,2.2mm)
    %%%% Warning: check controls
    .. controls (62.0mm,46.0mm) and (54.5mm,53.5mm) .. (53.0mm,54.8mm)
    ;
    %% path id="kvg:05b66-s7"
    %% path spec="m 50.75,55.75 c 4,8.75 7.18,24.67 1.75,38 -2.75,6.75 -7.75,1.25 -9.75,-2"
    \draw[/warp] (50.8mm,55.8mm)
    .. controls ++(4.0mm,8.8mm) and ++(5.4mm,-13.3mm) .. ++(1.8mm,38.0mm)
    .. controls ++(-2.8mm,6.8mm) and ++(2.0mm,3.2mm) .. ++(-9.8mm,-2.0mm)
    ;
    %% path id="kvg:05b66-s8"
    %% path spec="m 15.75,67.75 c 1.75,1 4.64,1.36 7.5,1 15.88,-2 44.43,-6.25 61.37,-5.5 2.5,0.11 4.72,0.25 6.39,1"
    \draw[/warp] (15.8mm,67.8mm)
    .. controls ++(1.8mm,1.0mm) and ++(-2.9mm,0.4mm) .. ++(7.5mm,1.0mm)
    .. controls ++(15.9mm,-2.0mm) and ++(-16.9mm,-0.8mm) .. ++(61.4mm,-5.5mm)
    .. controls ++(2.5mm,0.1mm) and ++(-1.7mm,-0.8mm) .. ++(6.4mm,1.0mm)
    ;
\end{scope}
        \end{scope}

		\fill[black]
			(0.5,6.5) node[right, scale=.7,name=inputlabel] {Input};
	\end{scope}

    \node (laba) [labelnode, left=5pt] at ($(inputlabel.west)$) {\textbf{(a)}};
    \node (labc) [labelnode] at ($(laba.mid)-(0,2)$) {\textbf{(c)}};
    \draw[tips, -{Latex[open,length=8pt]}] ($(laba.mid)+(15pt,-5pt)$) to [edge label'=\textbf{(b)}] ($(labc.mid)+(15pt,5pt)$);
    \node (labd) [labelnode, below left] at ($(d.mid)$) {\textbf{(d)}};
\begin{scope}[on background layer]
% Interlayer crossconnections
\draw (obstop) edge [thick, echoreg!65!black, arrows={-Latex[length=10pt, sep=-8pt]}, decoration={snake, segment length=2mm, amplitude=0.2mm}, decorate, double, in=90] (a);
\draw (obsbot) edge [thick, blue!70!black, arrows={-Latex[length=10pt, sep=-5pt]}, decoration={snake, segment length=2mm, amplitude=0.2mm}, double, decorate, out=90,in=90] (b);

\end{scope}

\end{tikzpicture}