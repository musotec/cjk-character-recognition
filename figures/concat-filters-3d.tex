% Under Creative Commons Attribution licence 3.0
% (http://creativecommons.org/licences/by/3.0)
% Author: Tom Fuller
% Adapted from TikZ Cuboid Computation by: Florian Lesaint


%\definecolor{col1}{RGB}{255,0,0}
%\definecolor{col2}{RGB}{0,255,0}
%\definecolor{col3}{RGB}{0,0,255}


\newcommand{\generateRefPoints}[7]{
    % 1 = square width (-2cm)   = to be transformed by perspective computation
    % 2 = square height (-2cm)  = given position at 0
    % 3 = center_x (0)
    % 4 = center_y (0)
    % 5 = range (8cm)
    % 6 = above (1.5cm)
    % 7 = below (2cm)
    %
    % // todo: take array of lengths relative to prior
    % 1 = len1
    % 2 = len2
    % 3 = len3

    \pgfmathparse{#4+#2}    % center - height = underview
    \pgfmathresult \let\yunder\pgfmathresult;

    \pgfmathparse{#6+#4}    % center - height = overview
    \pgfmathresult \let\yabove\pgfmathresult;

    \pgfmathparse{#5-#3}    % center - width = base (positive)
    \pgfmathresult \let\px\pgfmathresult;

    \pgfmathparse{-#2}
    \pgfmathresult \let\ybelow\pgfmathresult;


    %    above = y_dist(a1 - p1); below = y_dist(a2 - p1)


    %    phi = atan(above/px)
    %    s*90 = 180 - 2*phi; solve for s
    %    x` = s * (above/below)
    %    y` = s * sin(phi)


    \pgfmathparse{atan2(\yabove,\px)}
    \pgfmathresult \let\phi\pgfmathresult;
    \pgfmathparse{(180 - 2*\phi)/90}    % obtain the side using law of sines, 90 degree angle avoids a sine call
    \pgfmathresult \let\side\pgfmathresult;

    % compute the new x from the above/below; simplified sin(90) = 1
    \pgfmathparse{\side * (\yabove / \ybelow)}
    \pgfmathresult \let\xprime\pgfmathresult;

    % TODO: FIX x-ratio by getting the ratio of the known height to the distance half way between P2
    % TODO: and then use as known side length for obtaining ratio to halfway to P1
    \pgfmathparse{#2/#1}
    \pgfmathresult \let\xratio\pgfmathresult;

    % now get a ratio to the perspective point from the relative x
    \pgfmathparse{(scalar(\px/1cm) + \xratio*\xprime)/scalar(\px/1cm)}  % TODO: devise a way to remove the units... cmon
    \pgfmathresult \let\ratio\pgfmathresult;


    %% Vanishing points for perspective handling
    \coordinate (P1) at (-#5, #6); % left vanishing point (To pick)
    \coordinate (P2) at (#5, #6);  % right vanishing point (To pick)

    %% (A1) and (A2) defines the 2 central points of the cuboid
    % TODO: use the parameters for center x and y
    \coordinate (A1) at (0em,0cm); % central top point (To pick)
    \coordinate (A2) at (0em,-#2); % central bottom point (To pick)

    % the picture always fits in half the range to make even
    \coordinate (X1) at ($(P2)!.5!(A1)$);
    \coordinate (X2) at ($(P2)!.5!(A2)$);

    %% (A3) to (A8) are computed given a unique parameter (or 2) .8
    % You can vary .8 from 0 to 1 to change perspective on left side
    \coordinate (A3) at ($(P1)!\ratio!(A2)$); % To pick for perspective
    \coordinate (A4) at ($(P1)!\ratio!(A1)$);

    % y prime is not needed, but is computed as such from law of sines
%    \pgfmathparse{\side * sin(\phi)}
%    \pgfmathresult \let\yprime\pgfmathresult;
%
%    % Draw debug point (needs yprime uncommented)
%    \draw[fill=black!80!transparent] (\xprime, \yprime) circle (2em) node[color=white] {\tiny \xratio};
}

% Generate the vanishing points for this concatenation
% TODO: clean up arguments for initial point generation
% TODO: factor out concatenated cube generation
\generateRefPoints{2cm}{2cm}{0}{0}{8cm}{1.5cm};

%%% Vanishing points for perspective handling
%\coordinate (P1) at (-8cm, 1.5cm); % left vanishing point (To pick)
%\coordinate (P2) at (8cm, 1.5cm);  % right vanishing point (To pick)
%
%%% (A1) and (A2) defines the 2 central points of the cuboid
%\coordinate (A1) at (0em,0cm); % central top point (To pick)
%\coordinate (A2) at (0em,-2cm); % central bottom point (To pick)
%
%\coordinate (X1) at ($(P2)!.5!(A1)$);
%\coordinate (X2) at ($(P2)!.5!(A2)$);
%
%% (A3) to (A8) are computed given a unique parameter (or 2) .8
% You can vary .8 from 0 to 1 to change perspective on left side
%\coordinate (A3) at ($(P1)!.8346!(A2)$); % To pick for perspective
%\coordinate (A4) at ($(P1)!.8346!(A1)$);

% You can vary .8 from 0 to 1 to change perspective on right side
\coordinate (A7) at ($(P2)!.9375!(A2)$);
\coordinate (A8) at ($(P2)!.9375!(A1)$);

%% Automatically compute the last 2 points with intersections
\coordinate (A5) at
(intersection cs: first line={(A8) -- (P1)},
   second line={(A4) -- (P2)});
\coordinate (A6) at
(intersection cs: first line={(A7) -- (P1)},
   second line={(A3) -- (P2)});


\coordinate (B1) at ($(P2)!.995!(A8)$);
\coordinate (B2) at ($(P2)!.995!(A7)$);

\coordinate (B3) at ($(P2)!.995!(A6)$);
\coordinate (B4) at ($(P2)!.995!(A5)$);

\coordinate (B7) at ($(P2)!.775!(A2)$);
\coordinate (B8) at ($(P2)!.775!(A1)$);


\coordinate (B5) at
  (intersection cs: first line={(B8) -- (P1)},
            second line={(A5) -- (P2)});

\coordinate (B6) at
  (intersection cs: first line={(B7) -- (P1)},
            second line={(A6) -- (P2)});



\coordinate (C1) at ($(P2)!.995!(B8)$);
\coordinate (C2) at ($(P2)!.995!(B7)$);

\coordinate (C3) at ($(P2)!.995!(B6)$);
\coordinate (C4) at ($(P2)!.995!(B5)$);

\coordinate (C7) at ($(P2)!.625!(B2)$);
\coordinate (C8) at ($(P2)!.625!(B1)$);


\coordinate (C5) at
(intersection cs: first line={(C8) -- (P1)},
second line={(B5) -- (P2)});

\coordinate (C6) at
(intersection cs: first line={(C7) -- (P1)},
second line={(B6) -- (P2)});

%%%%%%%%%%%%%%%%%%%%%%
%%%%%%%%%%%%%%%%%%%%%% Start filling

\fill[col1!70!black] (C2) -- (C3) -- (C6) -- (C7) -- cycle; % bottom face
\fill[col1!50] (C3) -- (C4) -- (C5) -- (C6) -- cycle;  % far left face
\fill[col1!30,opacity=0.5] (C5) -- (C6) -- (C7) -- (C8) -- cycle; % far right face
\draw[thick,dashed] (C5) -- (C6);
\draw[thick,dashed] (C3) -- (C6);
\draw[thick,dashed] (C7) -- (C6);
\fill[col1,opacity=0.7] (C1) -- (C8) -- (C7) -- (C2) -- cycle; % face 1
\node at (barycentric cs:C1=1,C8=1,C7=1,C2=1) {}; % \tiny c1};
\fill[col1,opacity=0.3] (C1) -- (C2) -- (C3) -- (C4) -- cycle; % c2
\fill[col1,opacity=0.7] (C1) -- (C4) -- (C5) -- (C8) -- cycle; % c5
\node (t1) [anchor=north] at (barycentric cs:C1=1,C4=1,C5=1,C8=1) {};
\draw[thick] (C1) -- (C2);
\draw[thick] (C3) -- (C4);
\draw[thick] (C7) -- (C8);
\draw[thick] (C1) -- (C4);
\draw[thick] (C1) -- (C8);
\draw[thick] (C2) -- (C3);
\draw[thick] (C2) -- (C7);
\draw[thick] (C4) -- (C5);
\draw[thick] (C8) -- (C5);

%\fill[green!70!black] (B2) -- (B3) -- (B6) -- (B7) -- cycle;
%\fill[green!50!black] (B3) -- (B4) -- (B5) -- (B6) -- cycle;
%\fill[green!30] (B5) -- (B6) -- (B7) -- (B8) -- cycle; % far face

\fill[col2!70!black] (B2) -- (B3) -- (B6) -- (B7) -- cycle;
\fill[col2!50] (B3) -- (B4) -- (B5) -- (B6) -- cycle;
\fill[col2!30,opacity=0.5] (B5) -- (B6) -- (B7) -- (B8) -- cycle; % far right face

\draw[thick,dashed] (B5) -- (B6);
\draw[thick,dashed] (B3) -- (B6);
\draw[thick,dashed] (B7) -- (B6);

\fill[col2,opacity=0.7] (B1) -- (B8) -- (B7) -- (B2) -- cycle; % face 1
\node at (barycentric cs:B1=1,B8=1,B7=1,B2=1) {}; % \tiny b1};

\fill[col2,opacity=0.3] (B1) -- (B2) -- (B3) -- (B4) -- cycle; % b2
%\node at (barycentric cs:B1=1,B2=1,B3=1,B4=1) {\tiny b2};

\fill[col2,opacity=0.7] (B1) -- (B4) -- (B5) -- (B8) -- cycle; % f5
\node (t2) [anchor=north] at (barycentric cs:B1=1,B4=1,B5=1,B8=1) {};


\draw[thick] (B1) -- (B2);
\draw[thick] (B3) -- (B4);
\draw[thick] (B7) -- (B8);
\draw[thick] (B1) -- (B4);
\draw[thick] (B1) -- (B8);
\draw[thick] (B2) -- (B3);
\draw[thick] (B2) -- (B7);
\draw[thick] (B4) -- (B5);
\draw[thick] (B8) -- (B5);

%%%
%%%

%% Possibly draw back faces
\fill[col3!70!black,opacity=0.7] (A2) -- (A3) -- (A6) -- (A7) -- cycle; % face 6
%\node at (barycentric cs:A2=1,A3=1,A6=1,A7=1) {\tiny f6};

\fill[col3!50!black,opacity=0.7] (A3) -- (A4) -- (A5) -- (A6) -- cycle; % face 3
%\node at (barycentric cs:A3=1,A4=1,A5=1,A6=1) {\tiny f3};

\fill[col3!30,opacity=0.5] (A5) -- (A6) -- (A7) -- (A8) -- cycle; % face 4
%\node at (barycentric cs:A5=1,A6=1,A7=1,A8=1) {\tiny f4};

\draw[thick,dashed] (A5) -- (A6);
\draw[thick,dashed] (A3) -- (A6);
\draw[thick,dashed] (A7) -- (A6);


%% Possibly draw front faces

\fill[col3,opacity=0.7] (A1) -- (A8) -- (A7) -- (A2) -- cycle; % face 1

\fill[col3!70!black,opacity=0.7] (A1) -- (A2) -- (A3) -- (A4) -- cycle; % f2

\fill[col3,opacity=0.7] (A1) -- (A4) -- (A5) -- (A8) -- cycle; % f5

\node at (barycentric cs:A1=1,A8=1,A7=1,A2=1) {};%\tiny a1};

\node at (barycentric cs:A1=1,A2=1,A3=1,A4=1) {};%\tiny a2};

\node (t3) [anchor=north] at (barycentric cs:A1=1,A4=1,A5=1,A8=1) {};

%% Possibly draw front lines
\draw[thick] (A1) -- (A2);
\draw[thick] (A3) -- (A4);
\draw[thick] (A7) -- (A8);
\draw[thick] (A1) -- (A4);
\draw[thick] (A1) -- (A8);
\draw[thick] (A2) -- (A3);
\draw[thick] (A2) -- (A7);
\draw[thick] (A4) -- (A5);
\draw[thick] (A8) -- (A5);

% Draw debug points
%\foreach \i in {1,2,...,8} {
%    \draw[fill=black] (A\i) circle (0.25em) node[color=white] {\tiny \i};
%}
%
%\draw[fill=black] (P1) circle (0.1em) node[below] {\tiny p1};
%\draw[fill=black] (P2) circle (0.1em) node[below] {\tiny p2};
