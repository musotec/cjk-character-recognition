\tikzset {
    shape kanjiframe/.style= {
        color=black!40,
        draw,
        fill=white,
        ultra thick,
    }
}

\def\defkanjipos#1#2 {
    \tikzset {
        #1/.pic = {
            \begin{scope}[every path/.style={color=black!40}]
            \pgftransformscale{0.5}
            \draw[rounded corners=1pt, shape kanjiframe] (0,0) rectangle (1,1);
            #2
            \end{scope}
            % \fill[fill=black!40] (0,0) rectangle ++(0.25, 0.5);
        }
    }
}

\defkanjipos{all}{
    \fill (0,0) rectangle (1,1);
}

\defkanjipos{top}{
    \fill (0,1) rectangle (1,0.5);
}

\defkanjipos{bottom}{
    \fill (0,0) rectangle (1,0.5);
}

\defkanjipos{left}{
    \fill (0,0) rectangle (0.5,1);
}

\defkanjipos{right}{
    \fill (0.5,0) rectangle (1,1);
}

\defkanjipos{kamae}{
    \fill (0,0) rectangle (1/3,1);
    \fill (2/3,0) rectangle ++(1/3,1);
    \fill (0,0) rectangle (1,1/3);
    \fill (0,2/3) rectangle ++(1,1/3);
}

\defkanjipos{tare}{
    \fill (0,0) rectangle (1/3,1);
    \fill (0,1) rectangle (1,2/3);
}

\defkanjipos{nyou}{
    \fill (0,0) rectangle (1/3,1);
    \fill (0,1/3) rectangle (1,0);
}

\defkanjipos{none}{}

\tikz {
    \foreach \x / \name in {0/top,1/bottom,2/left,3/right,4/kamae,5/tare,6/nyou,7/all} {
        \node at (0.25+\x,0.7) {\x};
        \pic at (\x,0) {\name};
    }
}
